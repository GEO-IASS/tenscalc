\documentclass[11pt]{article}

% \usepackage{amsmath,amssymb,mathtools,amsfonts}
\usepackage{url} 
\usepackage{fullpage}
% \usepackage{setspace}      % \singlespace, \doublespace \onehalfspacing
% \usepackage{ifthen}      % \ifthenelse,\boolean,\newboolean,\setboolean
% \usepackage{mathptmx} % slightly more compressed font. 
\usepackage[T1]{fontenc}\usepackage[condensed,math]{kurier} % fancy font
% \usepackage{makeidx}  % to make a keyword index: \index{}
% \usepackage{showidx}  % prints index entries in the left margin (debug)
% \usepackage{needspace}     % \needspace{5\baselineskip} no page breaks for 5 lines
% \usepackage{mparhack} % correct Latex bug in \marginpar
% \usepackage{chemarr}  % arrows 4 chem: \xrightleftharpoons[]{} \xrightarrow{}
% \usepackage{listings} % source code printer for latex
% \lstset{language=Matlab}
% \lstset{basicstyle=\small,morekeywords={cvx_begin,cvx_end,variables,maximize,minimize,subject,to,linprog,quadprog,ones,optimset}}

%%%% Figure packages
% \usepackage{graphicx,psfrag}
% \usepackage{pstool}           % \psfrag for pdflatex -- preferable(?), not transparent 
% \usepackage{auto-pst-pdf}      % \psfrag & PStricks for pdflatex -- transparent!!!
% \usepackage[pdftex]{graphics}
% \usepackage{subfigure}         % \subfigure[a]{\includegraphics\ldots\label{fi:\ldots}}
% \usepackage{sidecaption}       % \sidecaption (to be placed inside figure env.
% \graphicspath{{./figuresdir1/}{./figuresdir2/}{./figuresdir3/}}
%%%%%%%%%%%%%%%%%%%%%%%%%% 

%%%% Bibliography packages (order is important)
% \usepackage{bibentry}% \nobibliography* \ldots  \bibentry{..} inserts a bib entry
% apparently incompatible with hyperef
% \makeatletter\let\NAT@parse\undefined\makeatother % enbl natbib with IEEE cls 
\usepackage[numbers,sort&compress,sectionbib]{natbib} % \cite,\citet,\citep,\ldots
\usepackage[colorlinks=true,linkcolor=blue,backref=page]{hyperref}
\renewcommand*{\backref}[1]{\small (cited in p.~#1)}
\usepackage[norefs,nocites]{refcheck} % options:norefs,nocites,msgs,chkunlbld
%%%%%%%%%%%%%%%%%%%%%%%%%% 

%%%%%%%%%%%%%% source code formatting for latex %%%%%%%%%%%%%%%%%%%%%%%%%%
\usepackage{listings}
%%%% Matlab language
\lstset{
  basicstyle={\footnotesize\tt},
  keywordstyle={\footnotesize\tt\color{blue}},
  commentstyle={\footnotesize\sl},
  frame=shadowbox,
  columns={[l]fixed},  % character alignment
  %columns={[l]fullflexible},
  basewidth={.5em,.4em}, 
  xleftmargin=3em, xrightmargin=3em,
  numbers=left, numberstyle=\tiny, numbersep=5pt,
  escapeinside={(*}{*)},
  numberblanklines=false,
  firstnumber=last,
}

\lstdefinelanguage[extended]{Matlab}[]{Matlab}{
  morekeywords={linprog,quadprog,ones,optimset,ode23s,expand},
  deletekeywords={hess}
}
%%%% my dialect of Matlab, based on Matlab without dialect
\lstdefinelanguage[TC]{Matlab}[extended]{Matlab}{
  morekeywords={Tvariable,parameter,Teye,Tones,Tzeros,Tconstant,gradient,norm2},
  morekeywords={csparse,declareSet,declareGet,declareCopy,declareComputation,compile2matlab,compile2C,
    begin_csparse,end_csparse,getFunction,setFunction,copyFunction}
}

%%%% set default language
\lstset{language=[TC]Matlab}
%%%%%%%%%%%%%%%%%%%%%%%%%%%%%%%%%%%%%%%%%%%%%%%%%%%%%%%%%%%%%%%%%%%%%%%%%

\usepackage[draft,fancythm,fancybb,morse]{jphmacros2e} 

%% Macros & options for this Document

\allowdisplaybreaks

\newcommand{\tprod}[2]{\,\,^{#1}\!\!*^{#2}}
\newcommand{\TC}{\texttt{TensCalcTools}}
\newcommand{\TTS}{\texttt{TTSTools}}
\newcommand{\CS}{\texttt{CSparse}}
\newcommand{\CMEX}{\texttt{cmexTools}}

\newcommand{\codesize}{\footnotesize}

\newcommand{\toidx}[1]{\index{\lstinline{#1}}}%\addcontentsline{toc}{subsubsection}{\lstinline{#1}}}

\newenvironment{command}[1]{\toidx{#1}\addcontentsline{toc}{subsection}{\lstinline{#1}}\paragraph*{\lstinline[basicstyle=\large,columns={[l]flexible}]{#1}}~\\\noindent\rule{\textwidth}{2pt}\\\vspace{-3ex}\codesize}{\vspace{-3ex}\rule{\textwidth}{1pt}\medskip\noindent}

% \makeindex

%% Start of Document

\title{\sc \TTS \\[1em]\Large \matlab{} Functions for Tensor-Valued Time-Series Processing}
\author{\jph}
\date{November 11, 2014}

\begin{document}                        \maketitle

\begin{abstract}
  This toolbox provides functions to process tensor-valued time
  series, including differentiations, integration, resampling,
  etc. These functions can be applied to numerical values or to \TC{}
  Symbolic Tensor-Valued Expressions (STVEs).
\end{abstract}

\tableofcontents

%\newpage

\section{Tensor-Valued Time Series}

\emph{Tensors} are essentially multi-dimensional arrays. Specifically,
an \emph{$\alpha$-index tensor} is an array in $\R^{n_1\times n_2\times\cdots\times n_\alpha}$ where
$\alpha$ is an integer in $\Z_{\ge 0}$. By convention, the case $\alpha=0$
corresponds to a \emph{scalar} in $\R$. We use the terminology
\emph{vector} and \emph{matrix} for the cases $\alpha=1$ and $\alpha=2$,
respectively. The integar $\alpha$ is called the \emph{index} of the tensor
and the vector of integers $[n_1,n_2,\dots,n_\alpha]$ (possibly empty for
$\alpha=0$) is called the \emph{dimension} of the tensor.

\medskip

A \emph{tensor-valued time series} (TTTS) is a sampled-based
representation of a time-varying tensor, i.e., a function $F:\R\to \R^{n_1\times
  n_2\times\cdots\times n_\alpha}$. A TTTS represents $F$ through a pair $(X,T)$ where
$T\in\R^{n_t}$ is a vector of sample times
and $X\in\R^{n_1\times n_2\times\cdots\times
  n_\alpha\times n_t}$ is an \emph{$\alpha+1$-index tensor} with the understanding that
$i\in\{1,2,\dots,n_t\}$,
\begin{align*}
  F(T_i)=\big[ X_{i_1,i_2,\dots,i_\alpha,i}\big]_{i_1=1,i_2=1,\dots,i_\alpha=1}^{i_1=n_1,i_2=n_2,\dots,i_\alpha=n_\alpha},
  \quad\forall i\in\{1,2,\dots,n_t\},
\end{align*}
i.e., the first $\alpha$ indices of $X$ represent the value of $F$ and the
last index represents time.

\newpage

\section{Functions provided by \TTS}

\begin{command}{tsDerivative}
\begin{lstlisting}
[dx,ts]=tsDerivative(x,ts)
\end{lstlisting}
\end{command}

This function returns a TTTS \lstinline{(dx,ts)} that represents the
time derivative of the input TTTS \lstinline{(x,ts)}. The output
sampling times \lstinline{ts} are equal to the input sampling times
\lstinline{ts} and therefore the size of derivatine \lstinline{dx} is
equal to the size of the input \lstinline{x}. The time derivative is
computed assuming that the input time-series is piecewise quadratic.

\begin{command}{tsDerivative2}
\begin{lstlisting}
[ddx,ts]=tsDerivative2(x,ts)
\end{lstlisting}
\end{command}

This function returns a TTTS \lstinline{(ddx,ts)} that represents the
second time derivative of the input TTTS \lstinline{(x,ts)}. The
output sampling times \lstinline{ts} are equal to the input sampling
times \lstinline{ts} and therefore the size of derivatine
\lstinline{dx} is equal to the size of the input \lstinline{x}. The
time derivatives are computed assuming that the input time-series is
piecewise quadratic.

\begin{command}{tsIntegral}
\begin{lstlisting}
y=tsIntegral(x,ts)
\end{lstlisting}
\end{command}

This function returns a tensor \lstinline{y} that represents the time
integral of the input TTTS \lstinline{(x,ts)}. The size of the
integral \lstinline{y} is equal to the size of the input \lstinline{x}
with the last (time) dimension removed. The integral is computed
assuming that the input time-series is piecewise quadratic.

\begin{command}{tsDot}
\begin{lstlisting}
[y,ts]=tsDot(x1,x2,ts)
\end{lstlisting}
\end{command}

This function returns a scalar-valued time-series \lstinline{(y,ts)} that
represents the dot product of two n-vector time-series
\lstinline{(x1,ts)} and \lstinline{(x2,ts)}:
\begin{align*}
  y=x1'.*x2
\end{align*}
The size of the output \lstinline{y} is equal to the size of
\lstinline{ts}.


\begin{command}{tsCross}
\begin{lstlisting}
[y,ts]=tsCross(x1,x2,ts)
\end{lstlisting}
\end{command}

This function returns a 3-vector time-series \lstinline{(y,ts)} that
represents the cross product of two 3-vector time-series
\lstinline{(x1,ts)} and \lstinline{(x2,ts)}:
\begin{align*}
  y=cross(x1,x2)
\end{align*}
The size of the output \lstinline{y} is equal to the size of the
inputs \lstinline{x1} and \lstinline{x2}.


\begin{command}{tsQdot}
\begin{lstlisting}
[y,ts]=tsQdot(q1,q2,ts)
\end{lstlisting}
\end{command}

This function returns a 4-vector time-series \lstinline{(y,ts)} that
represents the product of two quaternions \lstinline{(q1,ts)},
\lstinline{(q2,ts)}:
\begin{align*}
  y=q1\cdot q2
\end{align*}
The size of the output \lstinline{y} is equal to the size of the
inputs \lstinline{x1} and \lstinline{x2}. However, if either
\lstinline{(q1,ts)} or \lstinline{(q2,ts)} is a 3-vector time series,
then the corresponding input quaternion is assumed pure, but the output
quaternion is always a 4-vector time series.

\begin{command}{tsQdotStar}
\begin{lstlisting}
[y,ts]=tsQdotStar(q1,q2,ts)
\end{lstlisting}
\end{command}

This function returns a 4-vector time-series \lstinline{(y,ts)} that
represents the product of two quaternions \lstinline{(q1,ts)*},
\lstinline{(q2,ts)}:
\begin{align*}
  y=q1^*\cdot q2
\end{align*}
The size of the output \lstinline{y} is equal to the size of the
inputs \lstinline{x1} and \lstinline{x2}. However, if either
\lstinline{(q1,ts)} or \lstinline{(q2,ts)} is a 3-vector time series,
then the corresponding input quaternion is assumed pure, but the output
quaternion is always a 4-vector time series.

\begin{command}{tsRotation}
\begin{lstlisting}
[y,ts]=tsRotation(q,x,ts)
\end{lstlisting}
\end{command}

This function returns a 3-vector time-series \lstinline{(y,ts)} that
represents the rotation of a 3-vector time-series \lstinline{(x,ts)}
by a 4-vector time-series \lstinline{(q,ts)} representing a
quaternion:
\begin{align*}
  y=q\cdot x\cdot q^*
\end{align*}
The size of the output \lstinline{y} is equal to the
size of the input \lstinline{x}.

\begin{command}{tsRotationT}
\begin{lstlisting}
[y,ts]=tsRotationT(q,x,ts)
\end{lstlisting}
\end{command}

This function returns a 3-vector time-series \lstinline{(y,ts)} that
represents the rotation of a 3-vector time-series \lstinline{(x,ts)}
by a 4-vector time-series \lstinline{(q,ts)*} representing a
quaternion:
\begin{align*}
  y=q^* \cdot x\cdot q
\end{align*}
The size of the output \lstinline{y} is equal to the
size of the input \lstinline{x}.


% \bibliographystyle{ieeetr}
% \bibliographystyle{abbrvnat}
% \bibliography{strings,jph,crossrefs}

% \printindex

\end{document}

%%% Local Variables: 
%%% mode: latex
%%% eval: (tex-pdf-mode)  ; only for pdflatex
%%% TeX-master: t
%%% End: 
