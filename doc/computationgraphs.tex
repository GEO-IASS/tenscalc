\documentclass[11pt]{article}

\usepackage{etex} % extended latex (including increased number of registers needed by many packages)
% \usepackage{amsmath,amssymb,mathtools,amsfonts}
\usepackage{url} 
\usepackage{fullpage}
% \usepackage{setspace}      % \singlespace, \doublespace \onehalfspacing
% \usepackage{ifthen}      % \ifthenelse,\boolean,\newboolean,\setboolean
% \usepackage{mathptmx} % slightly more compressed font. 
\usepackage[T1]{fontenc}\usepackage[condensed,math]{kurier} % fancy font
% \usepackage{needspace}     % \needspace{5\baselineskip} or \Needspace{5\baselineskip} 
% \usepackage{mparhack} % correct Latex bug in \marginpar
% \usepackage{chemarr}  % arrows 4 chem: \xrightleftharpoons[]{} \xrightarrow{}
% \usepackage{listings} % source code printer for latex
% \lstset{language=Matlab}
% \lstset{basicstyle=\small,morekeywords={cvx_begin,cvx_end,variables,maximize,minimize,subject,to,linprog,quadprog,ones,optimset}}

%%%% Figure packages
% \usepackage{graphicx,psfrag}
% \usepackage{auto-pst-pdf}   % \psfrag & PStricks for pdflatex -- transparent buggy
% \usepackage[pdf]{pstricks}  % \psfrag for pdflatex (include of .eps sometimes leads to trouble) -- transparent!!!
% \usepackage{subfigure}      % \subfigure[a]{\includegraphics\ldots\label{fi:\ldots}}
% \usepackage{sidecaption}    % \sidecaption (to be placed inside figure env.
% \graphicspath{{./figuresdir1/}{./figuresdir2/}{./figuresdir3/}}
%%%%%%%%%%%%%%%%%%%%%%%%%% 

%%%% Bibliography packages (order is important)
% \usepackage{bibentry}% \nobibliography* \ldots  \bibentry{..} inserts a bib entry
% apparently incompatible with hyperef
% \makeatletter\let\NAT@parse\undefined\makeatother % enbl natbib with IEEE cls 
\usepackage[numbers,sort&compress,sectionbib]{natbib} % \cite,\citet,\citep,\ldots
\usepackage[colorlinks=true,linkcolor=blue,backref=page]{hyperref}
\renewcommand*{\backref}[1]{\small (cited in p.~#1)}
\usepackage[norefs,nocites]{refcheck} % options:norefs,nocites,msgs,chkunlbld
% \usepackage{makeidx} % to make a keyword index
% \usepackage{showidx} % prints index entries in the margin (debug) - NOT COMPATIBLE WITH hyperref
%%%%%%%%%%%%%%%%%%%%%%%%%% 

\usepackage[fancythm,fancybb,morse]{jphmacros2e} 
%\usepackage[draft,fancythm,fancybb,morse]{jphmacros2e} 

%% Macros & options for this Document

\allowdisplaybreaks

% \makeindex

%% Start of Document

\title{Computation graphs}
\author{\jph}
\date{June 13, 2016}

\begin{document}                        \maketitle


\section{Computation graph}

We are interested in describing a complex computation through a
dependency directed bipartite graph. One type of nodes corresponds to
multiple-input/multiple-output functions of the form
\begin{align}\label{eq:function}
  (y_1,y_2,\dots,y_m)=f(x_1,x_2,\dots,x_n)
\end{align}
and the other type of nodes correspond to variables that are either
inputs of outputs to the functions. The node corresponding to the
function \eqref{eq:function} has input edges from the variables
$x_1,x_2,\dots,x_n$ to $f$ and output edges from $f$ to the variables
$y_1,y_2,\dots,y_m$. 

\medskip

Function nodes without input edges are used to represent constant
variables. Variable nodes without input edges correspond to variables
that must be computed externally.

\medskip

The order of the inputs and outputs to the function
\eqref{eq:function} matters, so the digraphs must have ``ordered
edges.'' The variable nodes may have different types, which are
defined implicitly by the function they are connected to.

\section{File format}

The file representation of computational graphs needs to support very
large computations, which dictates the need for very efficient and
economical representations. Four files are used to represented the
computational graphs:
\begin{itemize}
\item \texttt{filename.cgc} describes the constant variables,
\item \texttt{filename.cgio} describes the functions used to interact
  with the computational graph,
\item \texttt{filename.cg} describes the graph structure,
\item \texttt{filename.cgs} describes symbolic names for the
  constants, variables, and functions.
\end{itemize}
All 4 files files start with a 16-bit ``magic'' integer that specifies
the format in which integers are stored in the file, followed by a
sequence of variable-length records. The 8-bit magic number can be
one of the following value:
\begin{itemize}
\item \texttt{16} - indices are stored as 16bit integers with the least
  significant byte first (little-endian byte ordering)
\item \texttt{32} - indices are 32bit integers  with the least
  significant byte first (little-endian byte ordering)
\item \texttt{64} - indices are 64bit integers with the least
  significant byte first (little-endian byte ordering)
\end{itemize}

\subsection{\texttt{.cgc} constants file}

Following the ``magic'' integer, this file consists of a sequence of
records, each corresponding to a constant that appears in the
computation graph. Each record is of the form:
\begin{enumerate}
\item 1 integer specifying the \emph{type} of the constant. All valid types of function should be registered.

\item 1 integer specifying the \emph{length} of the data stored in 8bit bytes.
\item Variable-length list of bytes with the \emph{data} representing
  the constant (in a format specific to the data type).
\end{enumerate}


\subsection{\texttt{.cg} graph file}

Following the ``magic'' integer, this file consists of a sequence of
records, each corresponding to a function node in the computation
graph. Each record is of the following form:
\begin{enumerate}
\item 1 integer specifying the \emph{type} of the function. All valid
  types of function should be registered.

  The special type \texttt{0} (zero) is used to denote a constant
  function that returns a single output equal to one of the functions
  in the \texttt{.cgc} constants file. In this case, the type integer
  is followed by
  \begin{enumerate}
  \item 0-based number of the record where the constant appears in the
    \texttt{.cgc} constants file
  \item 0-based integer specifying the output variable to the
    function.
  \end{enumerate}
  For this type of function node, the items below do not apply.

\item 1 integer specifying the number of input variables to the
  function. 

\item 1 integer specifying the number of output variables to the
  function.

\item Variable-length list of 0-based integers specifying the (appropriately
  ordered) input variables to the function. 

\item Variable-length list of 0-based integers specifying the
  (appropriately ordered) output variables to the function.
\end{enumerate}



\subsection{\texttt{.cgio} I/O file}

Following the ``magic'' integer, this file consists of a sequence of
records, each corresponding to one gateway function used to interact
with the computational graph. Each record is of the following form:
\begin{enumerate}
\item 1 integer specifying the \emph{type} of the function:
  \begin{itemize}
  \item \texttt{1} - gateway ``set'' function used set the value of
    variable nodes.
  \item \texttt{2} - gateway ``get'' function used to get the value of
    variable nodes.
  \item \texttt{3} - gateway ``copy'' function used to copy the values
    of variable nodes to other variable nodes.
  \end{itemize}
  
\item 1 integer specifying the number of variables to set/get/copy

\item Variable-length list of 0-based integers specifying the
  (appropriately ordered) variables to set, get, or copy-from (source)

\item Variable-length list of 0-based integers specifying the
  (appropriately ordered) variables to copy-to (destimation). Only for
  gateway ``copy'' functions.
\end{enumerate}

\subsection{\texttt{.cgs} symbolic name file}

Following the ``magic'' integer, this file consists of a sequence of
records, each corresponding to a symbolic name for a constant,
variable, or function. Each record is of the following form:
\begin{enumerate}
\item 1 integer specifying the type of the record:
  \begin{itemize}
  \item \texttt{1} - constant
  \item \texttt{2} - function node
  \item \texttt{3} - variable node
  \item \texttt{4} - io function
  \end{itemize}

\item 1 integer with the 0-based index of the
  constant/function/variable
\item 1 integer with number of characters in the symbolic name
\item 1 integer with number of characters in the
  constant/function/variable description
\item Variable-length string with the symbolic name
\item Variable-length string with the constant/function/variable
  description
\end{enumerate}
This file is mostly used for debug.

% \bibliographystyle{ieeetr}
% \bibliographystyle{abbrvnat}
% \bibliography{strings,jph,crossrefs}

% \printindex

\end{document}

%%% Local Variables: 
%%% mode: latex
%%% eval: (tex-pdf-mode)  ; only for pdflatex
%%% TeX-master: t
%%% End: 

