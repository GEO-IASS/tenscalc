\documentclass[11pt]{article}

\usepackage{etex} % extended latex (including increased number of registers needed by many packages)
%\usepackage{amsmath,amssymb,mathtools,amsfonts}
\usepackage{url} 
\usepackage{fullpage}
%\usepackage{setspace}      % \singlespace, \doublespace \onehalfspacing
%\usepackage{ifthen}      % \ifthenelse,\boolean,\newboolean,\setboolean
%\usepackage{mathptmx} % slightly more compressed font. 
\usepackage[T1]{fontenc}\usepackage[condensed,math]{kurier} % fancy font
%\usepackage{makeidx}  % to make a keyword index: \index{}
%\usepackage{showidx}  % prints index entries in the left margin (debug)
%\usepackage{needspace}     % \needspace{5\baselineskip} no page breaks for 5 lines
%\usepackage{mparhack} % correct Latex bug in \marginpar
%\usepackage{chemarr}  % arrows 4 chem: \xrightleftharpoons[]{} \xrightarrow{}
%\usepackage{listings} % source code printer for latex
%\lstset{language=Matlab}
%\lstset{basicstyle=\small,morekeywords={cvx_begin,cvx_end,variables,maximize,minimize,subject,to,linprog,quadprog,ones,optimset}}

%%%% Figure packages
%\usepackage{graphicx,psfrag}
%\usepackage{pstool}           % \psfrag for pdflatex -- preferable(?), not transparent 
%\usepackage{auto-pst-pdf}      % \psfrag & PStricks for pdflatex -- transparent!!!
%\usepackage{subfigure}         % \subfigure[a]{\includegraphics\ldots\label{fi:\ldots}}
%\usepackage{sidecaption}       % \sidecaption (to be placed inside figure env.
%\graphicspath{{./figuresdir1/}{./figuresdir2/}{./figuresdir3/}}
%%%%%%%%%%%%%%%%%%%%%%%%%%

%%%% Bibliography packages (order is important)
%\usepackage{bibentry}% \nobibliography* \ldots  \bibentry{..} inserts a bib entry
         % apparently incompatible with hyperef
%\makeatletter\let\NAT@parse\undefined\makeatother % enbl natbib with IEEE cls 
\usepackage[numbers,sort&compress,sectionbib]{natbib} % \cite,\citet,\citep,\ldots
\usepackage[colorlinks=true,linkcolor=blue,backref=page]{hyperref}
\renewcommand*{\backref}[1]{\small (cited in p.~#1)}
\usepackage[norefs,nocites]{refcheck} % options:norefs,nocites,msgs,chkunlbld
%%%%%%%%%%%%%%%%%%%%%%%%%%

\usepackage[fancythm,fancybb,morse]{jphmacros2e} 
%\usepackage[draft,fancythm,fancybb,morse]{jphmacros2e} 

%% Macros & options for this Document

\newcommand{\TC}{\texttt{TensCalcTools}}

\allowdisplaybreaks

%\makeindex

%% Start of Document

\title{Interior-Point Methods in \TC}
\author{\jph}
\date{April 6, 2017}

\begin{document}                        \maketitle

\begin{abstract}
  This document describes the optimization method used by \TC{} to
  solve constrained optimizations.
\end{abstract}

\section{Constrained Optimization}
\label{sec:ipm-o}

Our goal is to find a vector $u^*\in\scr{U}$ that solves the two coupled
optimizations
\begin{align}\label{eq:minimization}
  f(u^*)&=\min_{u\in\scr{U}} f(u), &
\end{align}
with
\begin{align*}
  \scr{U}&\eqdef\{u\in\R^{N}: F(u)\ge 0,\; G(u)=0\}, 
\end{align*}
for given functions $f:\R^{N}\in\R$, $F:\R^{N}\to\R^{M}$, and $G:\R^{N}\to\R^{K}$.


\subsection{Primal-dual method}

The following duality-like result provides the motivation for a
primal-dual-like method to solve the coupled minimizations in
\eqref{eq:minimization}. It provides a set of
conditions, involving an unconstrained optimization, that provide an
approximation to the solution of \eqref{eq:minimization}.
\begin{lemma}[Approximate equilibrium]\label{le:minima-gap}
  Suppose that we have found \emph{primal variables} $u\in\R^{N}$
  and \emph{dual variables} $\lambda\in\R^{M}$, $\nu\in\R^{K}$ that
  simultaneously satisfy all of the following
  conditions\footnote{Given a vector $x\in\R^n$ and a scalar $a\in\R$, we
    denote by $x\ge a$ the entry-wise ``greater than or equal to''
    comparison of each entry of $x$ with $a$.}
  \begin{subequations}
    \begin{align}
      \label{eq:equality-constraints}
      &G(u)=\mbf 0_K, \\
      \label{eq:inequality-constraints}
      &\lambda\;\dot\ge\;\mbf 0_M, & &F(u)\ge\mbf 0_M,\\
      \label{eq:central-points}
      &L_f(u,\lambda,\nu) =\min_{\bar u\in\R^{N}}\; L_f(\bar u,\lambda,\nu), 
    \end{align}
  \end{subequations}
  where $L_f(\bar u,\lambda,\nu)\eqdef f(\bar u)-\lambda F(\bar u)+\nu G(\bar u)$.  Then $u$ approximately
  satisfies \eqref{eq:minimization} in the sense that
  \begin{align}\label{eq:epsilon-equilibria-o}
    f(u)&\le\epsilon_f+\min_{\bar u\in\scr{U}} f(\bar u), &
    \epsilon_f&\eqdef \lambda F(u).
  \end{align}
  \frqed
\end{lemma}

\begin{proof-lemma}{\ref{le:minima-gap}}
  From the definitions of $L_f$, $\epsilon_f$ and using
  \eqref{eq:central-points}, we conclude that
  \begin{align*}
    f(u)-\epsilon_f& =\min_{\bar u\in\R^{N}}\; f(\bar u)-\lambda F(\bar u)+\nu G(\bar u).
  \end{align*}
  Therefore
  \begin{align*}
    f(u) =\epsilon_f+\min_{u\in\R^{N}}\; f(u)-\lambda F(u)+\nu G(u)
    \le \epsilon_f+\min_{u\in\scr{U}}\; f(u)-\lambda F(u)
    \le \epsilon_f+\max_{\lambda\ge0,\nu}\;\min_{u\in\scr{U}}\; f(u)-\lambda F(u),
  \end{align*}
  where the first inequality is a consequence of the fact that
  $\scr{U}\subset \R^N$ and \eqref{eq:equality-constraints}, whereas the
  second inequality is a consequence of the fact that maximizing over
  $\lambda\ge0$ will always lead to a larger value than what would be obtained
  for $\lambda\ge 0$, because of
  \eqref{eq:inequality-constraints}. Because of
  \eqref{eq:equality-constraints}, we further conclude that
  \begin{align*}
    f(u)&\le \epsilon_f+\max_{\lambda\ge0,\nu}\;\min_{u\in\scr{U}}\; f(u)-\lambda F(u)
    =\epsilon_f+\min_{\bar u\in\scr{U}}\; f(\bar u).
    \tag*\frqed
  \end{align*}
\end{proof-lemma}

\subsection{Interior-point primal-dual equilibria algorithm}

The method proposed consists of using Newton iterations to solve a
system of nonlinear equations on the primal variables $u\in\R^{N}$
and dual variables $\lambda\in\R^{M}$, $\nu\in\R^{K}$ introduced in
Lemma~\ref{le:minima-gap}. The specific system of equations
consists of:
\begin{enumerate}
\item the first-order optimality conditions for the unconstrained
  minimizations in \eqref{eq:central-points}\footnote{Given an integer
    $M$, we denote by $\mbf 0_M$ and by $\mbf 1_M$ the $M$-vectors
    with all entries equal to 0 and 1, respectively.}:
  \begin{align}\label{eq:optimality}
    \nabla L_f(u,\lambda,\nu)&=\mbf 0_{N}, &
  \end{align}
\item the equality conditions \eqref{eq:equality-constraints}; and
\item the equations\footnote{Given two vectors $x,y\in\R^n$ we denote by
    $x\odot y\in\R^n$ and by $x\oslash y\in\R^n$ the entry-wise product and
    division of the two vectors, respectively.}
  \begin{align}\label{eq:barrier}
    F(u)\odot\lambda&=\mu\mbf 1_{M},
  \end{align}
  for some $\mu>0$, which leads to
  \begin{align*}
    \epsilon_f&=M \mu.
  \end{align*}
\end{enumerate}
Since our goal is to find primal variables $u$ for which
\eqref{eq:epsilon-equilibria-o} holds with $\epsilon_f=0$, we shall make the
variable $\mu$ converge to zero as the Newton iterations
progresses. This is done in the context of an interior-point method,
meaning that all variables will be initialized so that the inequality
constraints \eqref{eq:inequality-constraints} hold \emph{strictly} and
the progression along the Newton direction at each iteration will be
selected so that these constraints are never violated.

\medskip

The specific steps of the algorithm that follows are based on the
primal-dual interior-point method for a single optimization, as
described in \cite{Vandenberghe2010}. To describe this algorithm, we
re-write \eqref{eq:optimality}, \eqref{eq:equality-constraints}, and
\eqref{eq:barrier} as
\begin{subequations}\label{eq:primal-dual}
  \begin{gather}
    \label{eq:primal-dual-equalities}
    \nabla L_f(u,\lambda,\nu)=\mbf 0_{N}, \qquad
    G(u)=\mbf 0 _{K}, \qquad
    \lambda\odot F(u)=\mu\mbf 1_{M}, 
  \end{gather}
and \eqref{eq:inequality-constraints} as
\begin{gather}
  \label{eq:primal-dual-inequalities}
  \lambda\ge \mbf 0_{M}, \qquad F(u)\ge \mbf 0_{M}.
\end{gather}
\end{subequations}

\clearpage

\begin{algorithm}[Primal-dual optimization]~\label{al:primal-dual-o}
\begin{steps}
\item Start with estimates $u_0,\lambda_0,\nu_0$ that satisfy the inequalities
  $\lambda_0>0$, $F(u_0)> 0$ in \eqref{eq:primal-dual-inequalities}
  and set $k=0$.

  It is often a good idea to start with a value for $u_0$ that
  satisfies the equality constraint $G(u_0)=0$, and
  \begin{align*}
    \lambda_0=\mu\mbf 1_{M}\oslash F(u_0),
  \end{align*}
  which guarantees that we initially have
  $\lambda_0\odot F(u_0)=\mu\mbf 1_{M}$.\joao{Starting with $\nu_0=0$ seems to be
    okay. Note that the value of $\nu$ only affects the matrix in the
    left-hand side of \eqref{eq:optimization-linearization-11} when
    $G(u)$ is nonlinear.}

\item Linearize the equations in \eqref{eq:primal-dual-equalities} around a
  current estimate $u_k,\lambda_k,\nu_k$, leading to
  \begin{align}\label{eq:optimization-linearization}
    \matt{
      \nabla_{uu} L_f(u_k,\lambda_k,\nu_k)&\nabla_{u\nu} L_f(u_k)&\nabla_{u\lambda} L_f(u_k)\\
      \nabla_u G(u_k) &0&0\\
      \diag(\lambda_k)\nabla_uF(u_k)&0&\diag[F(u_k)] }
    \matt{\Delta u\\\Delta\nu\\\Delta\lambda} =-\matt{
      \nabla_u L_f(u_k,\lambda_k,\nu_k)\\
      G(u_k)\\
      F(u_k)\odot\lambda_k-\mu\mbf 1_M }\notag\\
    \eqv
    \matt{
      \nabla_{uu} L_f(u_k,\lambda_k,\nu_k)&\nabla_u G(u_k)'&-\nabla_uF(u_k)'\\
      \nabla_u G(u_k) &0&0\\
      -\nabla_uF(u_k)&0&-\diag[F(u_k)\oslash \lambda_k] }
    \matt{\Delta u\\\Delta\nu\\\Delta\lambda} =-\matt{
      \nabla_u L_f(u_k,\lambda_k,\nu_k)\\
      G(u_k)\\
      -F(u_k)+\mu\mbf 1_M\oslash \lambda_k }
  \end{align}
  Since $F(u_k)>0$ and $\lambda_k>0$, we can solve this system of equations
  by first eliminating 
  \begin{subequations}\label{eq:optimization-linearization-1}
    \begin{align}
      \Delta\lambda=-\lambda_k-\diag[\lambda_k\oslash F(u_k)]\nabla_uF(u_k)\Delta u+\mu\mbf 1_M\oslash
      F(u_k),
    \end{align}
    which leads to\footnote{For numerical stability, one may solve
      instead
      \begin{multline}\label{eq:optimization-linearization-111}
        \matt{
          \nabla_{uu} L_f(u_k,\lambda_k,\nu_k)+\nabla_uF(u_k)' \diag[\lambda_k\oslash F(u_k)]\nabla_uF(u_k)+\epsilon I& \nabla_u G(u_k)'\\
          \nabla_u G(u_k) &-\epsilon I }
        \matt{\Delta u\\\Delta\nu} \\
        =-\matt{
          \nabla_u  f(u_k) +\nabla_u  G(u_k)'\nu_k-\mu\nabla_uF(u_k)'\big(\mbf 1_M\oslash F(u_k)\big)\\
          G(u_k) }
      \end{multline}
      for a small $\epsilon>0$. The matrix in the left-hand side has the top
      submatrix positive definite and the bottom one negative definite
      and is therefore better conditioned (see \tmpcite{Saunders 1996,
        Sect 4.2,MattingleyBoyd}). Nevertheless, upon convergence we
      still get the right-hand side zero, which still gives us
      \eqref{eq:primal-dual}. }
    \begin{multline}\label{eq:optimization-linearization-11}
      \matt{
        \nabla_{uu} L_f(u_k,\lambda_k,\nu_k)+\nabla_uF(u_k)' \diag[\lambda_k\oslash F(u_k)]\nabla_uF(u_k)& \nabla_u G(u_k)'\\
        \nabla_u G(u_k) &0 }
      \matt{\Delta u\\\Delta\nu} \\
      % =-\matt{
      %   \nabla_u L_f(u_k,\lambda_k,\nu_k)+\nabla_uF(u_k)' (\lambda_k-\mu\mbf 1_M\oslash F(u_k))\\
      %   G(u_k) }
      =-\matt{
        \nabla_u  f(u_k) +\nabla_u  G(u_k)'\nu_k-\mu\nabla_uF(u_k)'\big(\mbf 1_M\oslash F(u_k)\big)\\
        G(u_k) }.
    \end{multline}
  \end{subequations}
  Note however that starting by eliminating $\Delta\lambda$ may be undesirable
  when $\nabla_uF(u_k)$ has full rows, because in this case
  $\nabla_uF(u_k)' \diag[\lambda_k\oslash F(u_k)]\nabla_uF(u_k)$ will be a full
  matrix, which will ``destroy'' the computational advantages of any
  sparsity in $\nabla_{uu} L_f(u_k,\lambda_k,\nu_k)$. This is the case, e.g., when
  $F$ has constraints that depend on every (or most) elements of $u$.
  
\item \label{en:loop-o} Find the \emph{affine scaling direction}
  $\smatt{\Delta u_a'&\Delta\nu_a'&\Delta\lambda_a'}'$ by solving
  \eqref{eq:optimization-linearization-1}
  for $\mu=0$:
  \begin{align*}
    &\matt{
      \nabla_{uu} L_f(u_k,\lambda_k,\nu_k)+\nabla_uF(u_k)' \diag[\lambda_k\oslash F(u_k)]\nabla_uF(u_k)& \nabla_u G(u_k)'\\
      \nabla_u G(u_k) &0 }
    \matt{\Delta u\\\Delta\nu} \\&\hspace{.5\textwidth}
    =-\matt{
      \nabla_u  f(u_k) +\nabla_u  G(u_k)'\nu_k\\
      G(u_k) }\\
    &\Delta\lambda=-\lambda_k-\diag[\lambda_k\oslash F(u_k)]\nabla_uF(u_k)\Delta u.
  \end{align*}

\item Select scalings so that the inequalities in
  \eqref{eq:primal-dual-inequalities} would not be violated along the
  affine scaling direction:
  \begin{align*}
    \alpha_a&\eqdef\min\{\alpha_\mrm{primal},\alpha_\mrm{dual}\}, 
  \end{align*}
  where
  \begin{align*}
    \alpha_\mrm{primal}&\eqdef \max\big\{\alpha\in[0,1]:F(u_k+\alpha \Delta u_a)\ge 0\big\}, &
    \alpha_\mrm{dual}&\eqdef \max\big\{\alpha\in[0,1]:\lambda_k+\alpha \Delta\lambda_a\ge 0 \big\}
  \end{align*}
  and define the following estimate for the ``quality'' of the affine
  scaling direction
  \begin{align*}
    \sigma\eqdef\Big(\frac{F(u_k+\alpha_a \Delta u_a)'(\lambda_k+\alpha_a \Delta\lambda_a)}{F(u_k)'\lambda_k}\Big)^\delta,
  \end{align*}
  where $\delta$ is a parameter typically selected equal to 2 or 3. Note that
  the numerator $F(u_k+\alpha_a \Delta u_a)\odot(\lambda_k+\alpha_a
  \Delta\lambda_a)$ is the value one would obtain for $\lambda\odot F(u)$ by moving
  purely along the affine scaling directions. A small value for $\sigma$
  thus indicates that a significant reduction in $\mu$ is possible.

\item Find the \emph{search direction} $\smatt{\Delta
    u_s'&\Delta\nu_s'&\Delta\lambda_s'}'$ by solving
  \eqref{eq:optimization-linearization-1} for
  $\mu=\sigma\frac{F(u_k)'\lambda_k}{M}$\joao{Check with code.}:
  \begin{multline*}
    \matt{
      \nabla_{uu} L_f(u_k,\lambda_k,\nu_k)&\nabla_u G(u_k)'&-\nabla_uF(u_k)\\
      \nabla_u G(u_k) &0&0\\
      -\nabla_uF(u_k)&0&-\diag[F(u_k)\oslash \lambda_k] }
    \matt{\Delta u_s\\\Delta\nu_s\\\Delta\lambda_s} \\
    =-\matt{
      \nabla_u L_f(u_k,\lambda_k,\nu_k)\\
      G(u_k)\\
      -F(u_k)-{\color{blue}(\nabla_uF(u_k) \Delta u_a)\odot\Delta\lambda_a\oslash \lambda_k}+\mu\mbf 1_M\oslash \lambda_k },
  \end{multline*}
  where the (optional) blue term would come from a 2nd order expansion
  of the left-hand side of the last equality in
  \eqref{eq:primal-dual-equalities}. \joao{Need reference or removal
    of the blue term.}  Since $F(u_k)>0$ and $\lambda_k>0$, we can
  solve this system of equations by first eliminating
  \begin{subequations}\label{eq:optimization-linearization-2}
    \begin{align}
      \Delta\lambda_s= -\diag[\lambda_k\oslash F(u_k)]\nabla_uF(u_k)\Delta u_s
      -\lambda_k -{\color{blue}(\nabla_uF(u_k) \Delta u_a)\odot\Delta\lambda_a\oslash F(u_k)}
      +\mu\mbf 1_M\oslash
      F(u_k),
    \end{align}
    which leads to
    \begin{multline}
      \matt{
        \nabla_{uu} L_f(u_k,\lambda_k,\nu_k)+\nabla_uF(u_k)' \diag[\lambda_k\oslash F(u_k)]\nabla_uF(u_k)& \nabla_u G(u_k)'\\
        \nabla_u G(u_k) &0 }
      \matt{\Delta u_s\\\Delta\nu_s} \\
      % =-\matt{
      %   \nabla_u L_f(u_k,\lambda_k,\nu_k)+\nabla_uF(u_k)' (\lambda_k-\mu\mbf 1_M\oslash F(u_k))\\
      %   G(u_k) }
      =-\matt{
        \nabla_u  f(u_k) +\nabla_u  G(u_k)'\nu_k
        +\nabla_uF(u_k)'({\color{blue}(\nabla_uF(u_k) \Delta u_a)\odot\Delta\lambda_a\oslash F(u_k)})
        -\mu\nabla_uF(u_k)'\big(\mbf 1_M\oslash F(u_k)\big)\\
        G(u_k) }.
    \end{multline}
  \end{subequations}


\item Update the estimates along the search direction so that the
  inequalities in \eqref{eq:primal-dual-inequalities} hold
  strictly:
  \begin{align*}
    u_{k+1}&=u_k+\alpha_s\Delta u_s, &
    \nu_{k+1}&=\nu_k+\alpha_s \Delta \nu_s, &
    \lambda_{k+1}&=\lambda_k+\alpha_s \Delta \lambda_s
  \end{align*}
  where
  \begin{align*}
    \alpha_s&\eqdef\min\{\alpha_\mrm{primal},\alpha_\mrm{dual}\}, 
  \end{align*}
  and
  \begin{align}
    \alpha_\mrm{primal}&\eqdef \max\Big\{\alpha\in[0,1]:F(u_k+\frac{\alpha}{.99} \Delta u_s)\ge 0\Big\}, &
    \alpha_\mrm{dual}&\eqdef \max\Big\{\alpha\in[0,1]:\lambda_k+\frac{\alpha}{.99} \Delta\lambda_s\ge 0 \Big\}.
  \end{align}

\item Repeat from \ref{en:loop-o} with an incremented value for $k$ until
  \begin{align}\label{eq:termination}
    \|\nabla_u L_f(u_k,\lambda_k,\nu_k)\|&\le \epsilon, &
    \|G(u_k)\|&\le \epsilon_G, &
    \lambda_k 'F(u_k)&\le \epsilon_\mrm{gap}.
  \end{align}
  for sufficiently small tolerances $\epsilon$, $\epsilon_G$, $\epsilon_\mrm{gap}$. \frqed
\end{steps}
\end{algorithm}

When the function $L_f$ that appear in the unconstrained minimizations
in \eqref{eq:central-points} have a single stationary point that
corresponds to their global minimum, termination of the
Algorithm~\ref{al:primal-dual-o} guarantees that the assumptions of
Lemma~\ref{le:minima-gap} hold [up to the tolerances in
\eqref{eq:termination}] and we obtain the desired solution to
\eqref{eq:minimization}.

\medskip

The desired uniqueness of the stationary point holds, e.g., when the
function $f(u)$ is convex in $u$, $F(u)$ is concave in $u$, and $G(u)$
is linear in $u$. However, in practice the
Algorithm~\ref{al:primal-dual-o} can find solutions to
\eqref{eq:minimization} even when these convexity assumptions do not
hold. For problems for which one cannot be sure whether the
Algorithm~\ref{al:primal-dual-o} terminated at a global minimum of the
unconstrained problem, one may run several instances of the algorithm
with random initial conditions. Consistent results for the
optimizations across multiple initializations will provide an
indication that a global minimum has been found.


\begin{remark}[Smoothness]
  Algorithm~\ref{al:primal-dual-o} requires all the functions $f,F,G$ to
  be twice differentiable for the computation of the matrices that
  appear in \eqref{eq:optimization-linearization}. However, this does
  not preclude the use of this algorithm in many problems where these
  functions are not differentiable, because it is often possible to
  re-formulate non-smooth optimizations into smooth ones by
  appropriate transformation that often introduce additional
  optimization variables. Common examples of these transformations
  include the minimization of criteria involving $\ell_p$ norms, such as
  the ``non-differentiable $\ell_1$ optimization''
  \begin{align*}
    \min \big\{ \|A_{m\times n}x-b\|_{\ell_1}+\cdots: x\in\R^n,\dots \big\}
  \end{align*}
  which is equivalent to the following ``smooth optimization''
  \begin{align*}
    \min \big\{v'\mbf 1_m +\cdots:x\in\R^n,v\in\R^m, -v\le Ax-b\le v,\dots\big\};
  \end{align*}
  or the ``non-differentiable $\ell_2$ optimization''
  \begin{align*}
    \min \big\{\|A_{m\times n}x-b\|_{\ell_2}+\cdots: x\in\R^n,\dots\big\}
  \end{align*}
  \begin{align*}
    \min \big\{v +\cdots:x\in\R^n,v\ge0, v^2\ge (Ax-b)'(Ax-b),\dots \big\}.
  \end{align*}
  More examples of such transformations can be found, e.g., in
  \cite{GrantBoyd2008}.\joao{We should dig for more references.} \frqed
\end{remark}

\section{Interior-Point Method for Minimax Problems}
\label{sec:ipm}

Our goal is to find a pair
$(u^*,d^*)\in\scr{U}[d^*]\times\scr{D}[u^*]$ that simultaneously
solves the two coupled optimizations
\begin{align}\label{eq:equilibria}
  f(u^*,d^*)&=\min_{u\in\scr{U}[d^*]} f(u,d^*), &
  g(u^*,d^*)&=\min_{d\in\scr{D}[u^*]} g(u^*,d),
\end{align}
with
\begin{align*}
    \scr{U}[d]&\eqdef\{u\in\R^{N_u}: F_u(u,d)\ge 0,\; G_u(u,d)=0\}, \\
    \scr{D}[u]&\eqdef\{d\in\R^{N_d}: F_d(u,d)\ge 0,\; G_d(u,d)=0\},
\end{align*}
for given functions $f:\R^{N_u}\times\R^{N_d}\in\R$, $g:\R^{N_u}\times\R^{N_d}\in\R$,
$F_u:\R^{N_u}\times\R^{N_d}\to\R^{M_u}$, $F_d:\R^{N_u}\times\R^{N_d}\to\R^{M_d}$,
$G_u:\R^{N_u}\times\R^{N_d}\to\R^{K_u}$, $G_d:\R^{N_u}\times\R^{N_d}\to\R^{K_d}$.


\subsection{Primal-dual method}

The following duality-like result provides the motivation for a
primal-dual-like method to solve the coupled minimizations in
\eqref{eq:equilibria}. It provides a set of
conditions, involving two unconstrained optimizations, that provide an
approximation to the solution of \eqref{eq:equilibria}.
\begin{lemma}[Approximate equilibrium]\label{le:equilibria-gap}
  Suppose that we have found \emph{primal variables} $\hat
  u\in\R^{N_u},\hat d\in\R^{N_d}$ and \emph{dual variables}
  $\hat\lambda_{fu}\in\R^{M_u},\hat\lambda_{gd}\in\R^{M_d}$,
  $\hat\nu_{fu}\in\R^{K_u},\hat\nu_{gd}\in\R^{K_d}$ that simultaneously satisfy
  all of the following conditions\footnote{Given a vector $x\in\R^n$ and a scalar 
  $a\in\R$, we denote by $x\dot\ge a$ the entry-wise comparison of $x$ greater than 
  or equal to $a$.}
  \begin{subequations}
    \begin{align}
      \label{eq:equality-constraints-ud}
      &G_u(\hat u,\hat d)=0, &
      &G_d(\hat u,\hat d)=0, \\
      \label{eq:inequality-constraints-ud}
      &\hat\lambda_{fu}\dot\ge0, \qquad \hat\lambda_{gd} \dot\ge 0, & &F_u(\hat u,\hat d)\ge0,
      \qquad F_d(\hat u,\hat d)\ge0,\\
      \label{eq:central-points-ud}
      &L_f(\hat u,\hat d,\hat\lambda_{fu},\hat\nu_{fu}) =\min_{u\in\R^{N_u}}\; L_f(u,\hat d,\hat \lambda_{fu},\hat \nu_{fu}), &
      &L_g(\hat u,\hat d,\hat\lambda_{gd},\hat\nu_{gd}) =\min_{d\in\R^{N_d}}\; L_g(\hat u,d,\hat \lambda_{gd},\hat \nu_{gd})
    \end{align}
  \end{subequations}
  where 
  \begin{align*}
    &L_f(u,d,\lambda_{fu},\nu_{fu})\eqdef f(u,d)-\lambda_{fu} F_u(u,d)+\nu_{fu} G_u(u,d), \\
    &L_g(u,d,\lambda_{gd},\nu_{gd})\eqdef g(u,d)-\lambda_{gd} F_d(u,d)+\nu_{gd} G_d(u,d),  
     \qquad \forall u,d,\lambda,\nu.
  \end{align*}
  Then the pair $(\hat u,\hat d)$ approximately satisfies
  \eqref{eq:equilibria} in the sense that
  %is an $(\epsilon_f,\epsilon_g)$-equilibrium in the sense that
  \begin{align}\label{eq:epsilon-equilibria}
    f(\hat u,\hat d)&\le\epsilon_f+\min_{u\in\scr{U}[\hat d]} f(u,\hat d), &
    g(\hat u,\hat d)&\le\epsilon_g+\min_{d\in\scr{D}[\hat u]} g(\hat u,d),
  \end{align}
  with
  \begin{align*}
    \epsilon_f&\eqdef \hat\lambda_{fu} F_u(\hat u,\hat d), &
    \epsilon_g&\eqdef \hat\lambda_{gd} F_d(\hat u,\hat d).
    \tag*\frqed
  \end{align*}
\end{lemma}

\begin{proof-lemma}{\ref{le:equilibria-gap}}
  The proof is a direct consequence of the following
  sequence of inequalities that start from the equalities in
  \eqref{eq:central-points-ud}:
\begin{align*}
  f(\hat u,\hat d)-\epsilon_f&=L_f(\hat u,\hat d,\hat\lambda_{fu},\hat\nu_{fu})-\hat\nu_{fu}G_u(\hat u,\hat d) 
  =\min_{u\in\R^{N_u}}\; L_f(u,\hat d,\hat \lambda_{fu},\hat \nu_{fu}) - 0 \\
  &=\min_{u\in\R^{N_u}}\; f(u,\hat d)-\hat \lambda_{fu} F_u(u,\hat d)+\hat \nu_{fu} G_u(u,\hat d)\\
  &\le \max_{\lambda_{fu}\dot\ge0,\nu_{fu}}\min_{u\in\R^{N_u}}\;
  f(u,\hat d)- \lambda_{fu} F_u(u,\hat d)+ \nu_{fu} G_u(u,\hat d)\\
  &\le \max_{\lambda_{fu}\dot\ge0,\nu_{fu}}\min_{u\in\scr{U}[\hat d]}\;
  f(u,\hat d)-\lambda_{fu} F_u(u,\hat d)+\nu_{fu} G_u(u,\hat d)\\
  &= \min_{u\in\scr{U}[\hat d]}\; f(u,\hat d) \\
%
  g(\hat u,\hat d)-\epsilon_g
  &=L_g(\hat u,\hat d,\hat \lambda_{gd},\hat \nu_{gd})-\hat \nu_{gd}G_d(\hat u,\hat d) 
  =\min_{d\in\R^{N_d}}\;L_g(\hat u,d,\hat \lambda_{gd},\hat \nu_{gd}) - 0 \\
  &=\min_{d\in\R^{N_d}}\; g(\hat u,d)-\hat \lambda_{gd} F_d(\hat u,d)+\hat \nu_{gd} G_g(\hat u,d)\\
  &\le \max_{\lambda_{gd}\dot\ge0,\nu_{gd}}\min_{d\in\R^{N_d}}\;
  g(\hat u,d)-\lambda_{gd} F_d(\hat u,d)+\nu_{gd} G_g(\hat u,d)\\
  &\le \max_{\lambda_{gd}\dot\ge0,\nu_{gd}}\min_{d\in\scr{D}[\hat u]}\;
  g(\hat u,d)-\lambda_{gd} F_d(\hat u,d)+\nu_{gd} G_g(\hat u,d)\\
  &=\min_{d\in\scr{D}[\hat u]}\; g(\hat u,d)
  \tag*\frqed.
\end{align*}
\end{proof-lemma}

\subsection{Interior-point primal-dual equilibria algorithm}

The method proposed consists of using Newton iterations to solve a
system of nonlinear equations on the primal variables $\hat
u\in\R^{N_u},\hat d\in\R^{N_d}$ and dual variables
$\hat\lambda_{fu}\in\R^{M_u},\hat\lambda_{gd}\in\R^{M_d}$,
$\hat\nu_{fu}\in\R^{K_u},\hat\nu_{gd}\in\R^{K_d}$ introduced in
Lemma~\ref{le:equilibria-gap}. The specific system of equations
consists of:
\begin{enumerate}
\item the first-order optimality conditions for the unconstrained
  minimizations in \eqref{eq:central-points-ud}\footnote{Given an integer
    $M$, we denote by $\mbf 0_M$ and by $\mbf 1_M$ the $M$-vectors
    with all entries equal to 0 and 1, respectively.}:
  \begin{align}\label{eq:equilibrium}
    \nabla_u L_f(\hat u,\hat d,\hat\lambda_{fu},\hat\nu_{fu})&=\mbf 0_{N_u}, &
    \nabla_d L_g(\hat u,\hat d,\hat\lambda_{gd},\hat\nu_{gd})&=\mbf 0_{N_d}; &
  \end{align}
\item the equality conditions \eqref{eq:equality-constraints-ud}; and
\item the equations\footnote{Given two vectors $x,y\in\R^n$ we denote by
    $x\odot y\in\R^n$ and by $x\oslash y\in\R^n$ the entry-wise product and
    division of the two vectors, respectively.}
  \begin{align}\label{eq:barrier-ud}
    F_u(\hat u,\hat d)\odot\hat\lambda_{fu}&=\mu\mbf 1_{M_u}, &
    F_d(\hat u,\hat d)\odot\hat\lambda_{gd}&=\mu\mbf 1_{M_d}, &
  \end{align}
  for some $\mu>0$, which leads to
  \begin{align*}
    \epsilon_f&=M_u \mu, & \epsilon_g&=M_d \mu.
  \end{align*}
\end{enumerate}
Since our goal is to find primal variables $\hat u,\hat d$ for which
\eqref{eq:epsilon-equilibria} holds with $\epsilon_f=\epsilon_g=0$, we shall make
the variable $\mu$ converge to zero as the Newton iterations
progress. This is done in the context of an interior-point method,
meaning that all variables will be initialized so that the inequality
constraints \eqref{eq:inequality-constraints-ud} hold and the progression
along the Newton direction at each iteration will be selected so that
these constraints are never violated.

\medskip

The specific steps of the algorithm that follows are inpired by the
primal-dual interior-point method for a single optimization, as
described in \cite{Vandenberghe2010}. To describe this algorithm, we
define
\begin{align*}
  z&\eqdef\matt{\hat u\\\hat d}, &
  \lambda&\eqdef\matt{\hat\lambda_{fu}\\\hat\lambda_{gd}}, &
  \nu&\eqdef\matt{\hat\nu_{fu}\\\hat\nu_{gd}}, &
  G(z)&\eqdef\matt{G_u(\hat u,\hat d)\\ G_d(\hat u,\hat d)}, &
  F(z)&\eqdef\matt{F_u(\hat u,\hat d)\\ F_d(\hat u,\hat d)}, &
\end{align*}
which allow us to re-write \eqref{eq:equilibrium},
\eqref{eq:equality-constraints-ud}, and \eqref{eq:barrier-ud} as
\begin{subequations}
  \begin{gather}
    \label{eq:primal-dual-ud-equalities}
    \nabla_u L_f(z,\lambda,\nu)=\mbf 0_{N_u}, \qquad
    \nabla_d L_g(z,\lambda,\nu)=\mbf 0_{N_d}, \qquad
    G(z)=\mbf 0 _{K_u+K_d}, \qquad
    \lambda\odot F(z)=\mu\mbf 1_{M_u+M_d}, 
  \end{gather}
and \eqref{eq:inequality-constraints-ud} as
\begin{gather}
  \label{eq:primal-dual-ud-inequalities}
  \lambda\ge \mbf 0_{M_u+M_d}, \qquad F(z)\ge \mbf 0_{M_u+M_d}.
\end{gather}
\end{subequations}

\clearpage

\begin{algorithm}[Primal-dual optimization]~\label{al:primal-dual}
\begin{steps}
\item Start with estimates $z_0,\lambda_0,\nu_0$ that satisfy the inequalities
  $\lambda_0\ge 0$, $F(z_0)\ge 0$ in \eqref{eq:primal-dual-ud-inequalities} and set $k=0$.

  It is often a good idea to start with a value for $z_0$ that
  satisfies the equality constraint $G(z_0)=0$, and
  \begin{align*}
    \lambda_0=\mu\mbf 1_{M_u+M_d}\oslash F(z_0),
  \end{align*}
  which guarantees that we initially have
  $\lambda_0\odot F(u_0)=\mu\mbf 1_{M_u+M_d}$.\joao{Starting with
    $\nu_0=0$ seems to be okay. Note that the value of $\nu$ only affects
    the matrix in the left-hand side of
    \eqref{eq:equilibrium-linearization-31} when $G(u)$ is nonlinear.}


\item Linearize the equations in \eqref{eq:primal-dual-ud-equalities} around a
  current estimate $z_k,\lambda_k,\nu_k$, leading to
  % \begin{align*}
  %   \matt{
  %     \nabla_{uz} L_f(z_k,\lambda_k,\nu_k)&\nabla_{u\nu} L_f(z_k)&\nabla_{u\lambda} L_f(z_k)\\
  %     \nabla_{dz} L_g(z_k,\lambda_k,\nu_k)&\nabla_{d\nu} L_g(z_k)&\nabla_{d\lambda} L_g(z_k)\\
  %     \nabla_z G(z_k) &0&0\\
  %     \diag(\lambda_k)\nabla_zF(z_k)&0&\diag[F(z_k)] }
  %   \matt{\Delta z\\\Delta\nu\\\Delta\lambda} =-\matt{
  %     \nabla_u L_f(z_k,\lambda_k,\nu_k)\\
  %     \nabla_d L_g(z_k,\lambda_k,\nu_k) \\
  %     G(z_k)\\
  %     F(z_k)\odot\lambda_k-\mu\mbf 1 }
  % \end{align*}
  % or equivalently\joao{This equation may not be needed.}
  \begin{align}\label{eq:equilibrium-linearization}
    \matt{
      \nabla_{uz} L_f(z_k,\lambda_k,\nu_k)&\nabla_{u\nu} L_f(z_k)&\nabla_{u\lambda} L_f(z_k)\\
      \nabla_{dz} L_g(z_k,\lambda_k,\nu_k)&\nabla_{d\nu} L_g(z_k)&\nabla_{d\lambda} L_g(z_k)\\
      \nabla_z G(z_k) &0&0\\
      -\nabla_zF(z_k)&0&-\diag[F(z_k)\oslash \lambda_k] }
    \matt{\Delta z\\\Delta\nu\\\Delta\lambda} =-\matt{
      \nabla_u L_f(z_k,\lambda_k,\nu_k)\\
      \nabla_d L_g(z_k,\lambda_k,\nu_k) \\
      G(z_k)\\
      -F(z_k)+\mu \mbf 1\oslash \lambda_k }.
  \end{align}
  Since the vectors $F(z_k), \lambda_k$ have positive entries, we can
  solve this system of equations by first eliminating
  \begin{subequations}\label{eq:equilibrium-linearization-3}
    \begin{multline}
      \nabla_z F(z_k)\Delta z
      +\diag[F(z_k)\oslash\lambda_k]\Delta\lambda
      =-F(z_k)+\mu \mbf 1\oslash \lambda_k \eqv\\
      \Delta\lambda
      =-\lambda_k
      -\diag[\lambda_k\oslash F(z_k)]\nabla_z F(z_k)\Delta z 
      +\mu \mbf 1\oslash F(z_k)
    \end{multline}
    which leads to
    \begin{multline}\label{eq:equilibrium-linearization-31}
      % \matt{
      %   \nabla_{uz} L_f(z_k,\lambda_k,\nu_k)&\nabla_{u\nu} L_f(z_k)\\
      %   \nabla_{dz} L_g(z_k,\lambda_k,\nu_k)&\nabla_{d\nu} L_g(z_k)\\
      %   \nabla_z G(z_k) &0}
      % \matt{\Delta z\\\Delta\nu}
      % +\matt{
      %   \nabla_{u\lambda} L_f(z_k)\\
      %   \nabla_{d\lambda} L_g(z_k)\\
      %   0}
      % \Delta\lambda
      % =-\matt{
      %   \nabla_u L_f(z_k,\lambda_k,\nu_k)\\
      %   \nabla_d L_g(z_k,\lambda_k,\nu_k) \\
      %   G(z_k)}
      % \eqv \\
      \matt{
        \nabla_{uz} L_f(z_k,\lambda_k,\nu_k)&\nabla_{u\nu} L_f(z_k)\\
        \nabla_{dz} L_g(z_k,\lambda_k,\nu_k)&\nabla_{d\nu} L_g(z_k)\\
        \nabla_z G(z_k) &0}
      \matt{\Delta z\\\Delta\nu}
      =-\matt{
        \nabla_u L_f(z_k,\lambda_k,\nu_k)+\nabla_{u\lambda} L_f(z_k)\Delta\lambda\\
        \nabla_d L_g(z_k,\lambda_k,\nu_k)+\nabla_{d\lambda} L_g(z_k)\Delta\lambda\\
        G(z_k)}
      % \eqv \\
      % \matt{
      %   \nabla_{uz} L_f(z_k,\lambda_k,\nu_k)&\nabla_{u\nu} L_f(z_k)\\
      %   \nabla_{dz} L_g(z_k,\lambda_k,\nu_k)&\nabla_{d\nu} L_g(z_k)\\
      %   \nabla_z G(z_k) &0}
      % \matt{\Delta z\\\Delta\nu}
      % =-\matt{
      %   \nabla_u L_f(z_k,\lambda_k,\nu_k)-\nabla_{u\lambda} L_f(z_k)\lambda_k
      %   -\nabla_{u\lambda} L_f(z_k)\diag[\lambda_k\oslash F(z_k)]\nabla_z F(z_k)\Delta z 
      %   +\mu \nabla_{u\lambda} L_f(z_k)\big(\mbf 1\oslash F(z_k)\big)\\
      %   \nabla_d L_g(z_k,\lambda_k,\nu_k)-\nabla_{d\lambda} L_g(z_k)\lambda_k
      %   -\nabla_{d\lambda} L_g(z_k)\diag[\lambda_k\oslash F(z_k)]\nabla_z F(z_k)\Delta z 
      %   +\mu \nabla_{d\lambda} L_g(z_k)\big(\mbf 1\oslash F(z_k)\big)\\
      %   G(z_k)}
      \eqv \\
      \matt{
        \nabla_{uz} L_f(z_k,\lambda_k,\nu_k) -\nabla_{u\lambda} L_f(z_k)\diag[\lambda_k\oslash F(z_k)]\nabla_z F(z_k)&\nabla_{u\nu} L_f(z_k)\\
        \nabla_{dz} L_g(z_k,\lambda_k,\nu_k) -\nabla_{d\lambda} L_g(z_k)\diag[\lambda_k\oslash F(z_k)]\nabla_z F(z_k)&\nabla_{d\nu} L_g(z_k)\\
        \nabla_z G(z_k) &0}
      \matt{\Delta z\\\Delta\nu}\\
      =-\matt{
        \nabla_u L_f(z_k,\lambda_k,\nu_k)-\nabla_{u\lambda} L_f(z_k)\lambda_k
        +\mu \nabla_{u\lambda} L_f(z_k)\big(\mbf 1\oslash F(z_k)\big)\\
        \nabla_d L_g(z_k,\lambda_k,\nu_k)-\nabla_{d\lambda} L_g(z_k)\lambda_k
        +\mu \nabla_{d\lambda} L_g(z_k)\big(\mbf 1\oslash F(z_k)\big)\\
        G(z_k)}
      % \eqv \\
      % \matt{
      %   \nabla_{uz} L_f(z_k,\lambda_k,\nu_k) -\nabla_{u\lambda} L_f(z_k)\diag[\lambda_k\oslash F(z_k)]\nabla_z F(z_k)&\nabla_{u\nu} L_f(z_k)\\
      %   \nabla_{dz} L_g(z_k,\lambda_k,\nu_k) -\nabla_{d\lambda} L_g(z_k)\diag[\lambda_k\oslash F(z_k)]\nabla_z F(z_k)&\nabla_{d\nu} L_g(z_k)\\
      %   \nabla_z G(z_k) &0}
      % \matt{\Delta z\\\Delta\nu}\\
      % =-\matt{
      %   \nabla_u \big(L_f(z_k,\lambda_k,\nu_k)-\nabla_{\lambda} L_f(z_k)\lambda_k\big)
      %   +\mu \nabla_{u\lambda} L_f(z_k)\big(\mbf 1\oslash F(z_k)\big)\\
      %   \nabla_d \big(L_g(z_k,\lambda_k,\nu_k)-\nabla_{\lambda} L_g(z_k)\lambda_k\big)
      %   +\mu \nabla_{d\lambda} L_g(z_k)\big(\mbf 1\oslash F(z_k)\big)\\
      %   G(z_k)}
      % \eqv \\
      % \matt{
      %   \nabla_{uz} L_f(z_k,\lambda_k,\nu_k) -\nabla_{u\lambda} L_f(z_k)\diag[\lambda_k\oslash F(z_k)]\nabla_z F(z_k)&\nabla_{u\nu} L_f(z_k)\\
      %   \nabla_{dz} L_g(z_k,\lambda_k,\nu_k) -\nabla_{d\lambda} L_g(z_k)\diag[\lambda_k\oslash F(z_k)]\nabla_z F(z_k)&\nabla_{d\nu} L_g(z_k)\\
      %   \nabla_z G(z_k) &0}
      % \matt{\Delta z\\\Delta\nu}\\
      % =-\matt{
      %   \nabla_u \big(f(z_k)+(\nu_{fu})_kG_u(z_k)\big)
      %   +\mu \nabla_{u\lambda} L_f(z_k)\big(\mbf 1\oslash F(z_k)\big)\\
      %   \nabla_d \big(g(z_k)+(\nu_{gd})_kG_d(z_k)\big)
      %   +\mu \nabla_{d\lambda} L_g(z_k)\big(\mbf 1\oslash F(z_k)\big)\\
      %   G(z_k)}
      \eqv \\
      \matt{
        \nabla_{uz} L_f(z_k,\lambda_k,\nu_k) -\nabla_{u\lambda} L_f(z_k)\diag[\lambda_k\oslash F(z_k)]\nabla_z F(z_k)&\nabla_{u\nu} L_f(z_k)\\
        \nabla_{dz} L_g(z_k,\lambda_k,\nu_k) -\nabla_{d\lambda} L_g(z_k)\diag[\lambda_k\oslash F(z_k)]\nabla_z F(z_k)&\nabla_{d\nu} L_g(z_k)\\
        \nabla_z G(z_k) &0}
      \matt{\Delta z\\\Delta\nu}\\
      =-\matt{
        \nabla_u f(z_k)+\nabla_u \big((\nu_{fu})_k G_u(z_k)\big)
        +\mu \nabla_{u\lambda} L_f(z_k)\big(\mbf 1\oslash F(z_k)\big)\\
        \nabla_d g(z_k)+\nabla_d \big((\nu_{gd})_k G_d(z_k)\big)
        +\mu \nabla_{d\lambda} L_g(z_k)\big(\mbf 1\oslash F(z_k)\big)\\
        G(z_k)}
    \end{multline}
  \end{subequations}


\item \label{en:loop} Find the \emph{affine scaling direction} $\smatt{\Delta
    z_a'&\Delta\nu_a'&\Delta\lambda_a'}'$ by solving \eqref{eq:equilibrium-linearization-3}
  for $\mu=0$:
  \begin{align*}
      &\matt{
        \nabla_{uz} L_f(z_k,\lambda_k,\nu_k) -\nabla_{u\lambda} L_f(z_k)\diag[\lambda_k\oslash F(z_k)]\nabla_z F(z_k)&\nabla_{u\nu} L_f(z_k)\\
        \nabla_{dz} L_g(z_k,\lambda_k,\nu_k) -\nabla_{d\lambda} L_g(z_k)\diag[\lambda_k\oslash F(z_k)]\nabla_z F(z_k)&\nabla_{d\nu} L_g(z_k)\\
        \nabla_z G(z_k) &0}
      \matt{\Delta z_a\\\Delta\nu_a}\\\\&\hspace{.5\textwidth}
      =-\matt{
        \nabla_u f(z_k)+\nabla_u \big((\nu_{fu})_kG_u(z_k)\big)\\
        \nabla_d g(z_k)+\nabla_d \big((\nu_{gd})_kG_d(z_k)\big)\\
        G(z_k)}\\
    &\Delta\lambda_a
    =-\lambda_k
    -\diag[\lambda_k\oslash F(z_k)]\nabla_z F(z_k)\Delta z_a.
  \end{align*}
\item Select scalings so that the inequalities in
  \eqref{eq:primal-dual-ud-inequalities} would not be violated along the
  affine scaling direction:
  \begin{align*}
    \alpha_a&\eqdef\min\{\alpha_\mrm{primal},\alpha_\mrm{dual}\}, 
  \end{align*}
  where
  \begin{align}
    \alpha_\mrm{primal}&\eqdef \max\big\{\alpha\in[0,1]:F(z_k+\alpha \Delta z_a)\ge 0\big\}, &
    \alpha_\mrm{dual}&\eqdef \max\big\{\alpha\in[0,1]:\lambda_k+\alpha \Delta\lambda_a\ge 0 \big\}
  \end{align}
  and define the following estimate for the ``quality'' of the affine
  scaling direction
  \begin{align*}
    \sigma\eqdef\Big(\frac{F(z_k+\alpha_\mrm{primal} \Delta z_a)'(\lambda_k+\alpha_\mrm{dual} \Delta\lambda_a)}{F(z_k)'\lambda_k}\Big)^\delta,
  \end{align*}
  where $\delta$ is a parameter typically selected equal to 2 or
  3. Note that the numerator $F(z_k+\alpha_a \Delta
  z_a)'(\lambda_k+\alpha_a \Delta\lambda_a)$ is the value one
  would obtain for $\lambda' F(z)$ by moving purely along the
  affine scaling directions. A small value for $\sigma$ thus indicates
  that a significant reduction in $\mu$ is possible.

\item Find the \emph{search direction} $\smatt{\Delta
    z_s'&\Delta\nu_s'&\Delta\lambda_s'}'$ by solving
  \eqref{eq:equilibrium-linearization-3} for
  $\mu=\sigma\frac{F(z_k)\odot\lambda_k}{M_u+M_d}$:
  \begin{multline*}
    \matt{
      \nabla_{uz} L_f(z_k,\lambda_k,\nu_k)&\nabla_{u\nu} L_f(z_k)&\nabla_{u\lambda} L_f(z_k)\\
      \nabla_{dz} L_g(z_k,\lambda_k,\nu_k)&\nabla_{d\nu} L_g(z_k)&\nabla_{d\lambda} L_g(z_k)\\
      \nabla_z G(z_k) &0&0\\
      -\nabla_zF(z_k)&0&-\diag[F(z_k)\oslash \lambda_k] }
    \matt{\Delta z_s\\\Delta\nu_s\\\Delta\lambda_s} \\
    =-\matt{
      \nabla_u L_f(z_k,\lambda_k,\nu_k)\\
      \nabla_d L_g(z_k,\lambda_k,\nu_k) \\
      G(z_k)\\
      -F(z_k)-{\color{blue}(\nabla_zF(z_k) \Delta z_a)\odot\Delta\lambda_a\oslash \lambda_k}+\mu \mbf 1\oslash \lambda_k },
  \end{multline*}
  where the (optional) blue term would come from a 2nd order expansion
  of the left-hand side of the last equality in
  \eqref{eq:primal-dual-ud-equalities}. \joao{Need reference or removal
    of the blue term.}
%
  Since the vectors $F(z_k), \lambda_k$ have positive entries, we can
  solve this system of equations by first eliminating
    \begin{multline*}
      \nabla_z F(z_k)\Delta z_s
      +\diag[F(z_k)\oslash\lambda_k]\Delta\lambda_s
      =-F(z_k)-{\color{blue}(\nabla_zF(z_k) \Delta z_a)\odot\Delta\lambda_a\oslash \lambda_k}+\mu \mbf 1\oslash \lambda_k \eqv\\
      \Delta\lambda_s
      =-\lambda_k
      -{\color{blue}(\nabla_zF(z_k) \Delta z_a)\odot\Delta\lambda_a\oslash F(z_k)}
      -\diag[\lambda_k\oslash F(z_k)]\nabla_z F(z_k)\Delta z_s 
      +\mu \mbf 1\oslash F(z_k)
    \end{multline*}
    which leads to
    \begin{multline*}
      % \matt{
      %   \nabla_{uz} L_f(z_k,\lambda_k,\nu_k)&\nabla_{u\nu} L_f(z_k)\\
      %   \nabla_{dz} L_g(z_k,\lambda_k,\nu_k)&\nabla_{d\nu} L_g(z_k)\\
      %   \nabla_z G(z_k) &0}
      % \matt{\Delta z_s\\\Delta\nu_s}
      % +\matt{
      %   \nabla_{u\lambda} L_f(z_k)\\
      %   \nabla_{d\lambda} L_g(z_k)\\
      %   0}
      % \Delta\lambda_s
      % =-\matt{
      %   \nabla_u L_f(z_k,\lambda_k,\nu_k)\\
      %   \nabla_d L_g(z_k,\lambda_k,\nu_k) \\
      %   G(z_k)}
      % \eqv \\
      \matt{
        \nabla_{uz} L_f(z_k,\lambda_k,\nu_k)&\nabla_{u\nu} L_f(z_k)\\
        \nabla_{dz} L_g(z_k,\lambda_k,\nu_k)&\nabla_{d\nu} L_g(z_k)\\
        \nabla_z G(z_k) &0}
      \matt{\Delta z_s\\\Delta\nu_s}
      =-\matt{
        \nabla_u L_f(z_k,\lambda_k,\nu_k)+\nabla_{u\lambda} L_f(z_k)\Delta\lambda_s\\
        \nabla_d L_g(z_k,\lambda_k,\nu_k)+\nabla_{d\lambda} L_g(z_k)\Delta\lambda_s\\
        G(z_k)}
%      \eqv \\
%       \matt{
%         \nabla_{uz} L_f(z_k,\lambda_k,\nu_k)&\nabla_{u\nu} L_f(z_k)\\
%         \nabla_{dz} L_g(z_k,\lambda_k,\nu_k)&\nabla_{d\nu} L_g(z_k)\\
%         \nabla_z G(z_k) &0}
%       \matt{\Delta z_s\\\Delta\nu_s}\\
%       =-\matt{
%         \nabla_u L_f(z_k,\lambda_k,\nu_k)
% -\nabla_{u\lambda} L_f(z_k)\lambda_k
%       -{\color{blue}\nabla_{u\lambda} L_f(z_k)\big((\nabla_zF(z_k) \Delta z_a)\odot\Delta\lambda_a\oslash F(z_k)\big)}
%       -\nabla_{u\lambda} L_f(z_k)\diag[\lambda_k\oslash F(z_k)]\nabla_z F(z_k)\Delta z_s 
%       +\mu \nabla_{u\lambda} L_f(z_k)\big(\mbf 1\oslash F(z_k)\big)\\
%         \nabla_d L_g(z_k,\lambda_k,\nu_k)
% -\nabla_{d\lambda} L_g(z_k)\lambda_k
%       -{\color{blue}\nabla_{d\lambda} L_g(z_k)\big((\nabla_zF(z_k) \Delta z_a)\odot\Delta\lambda_a\oslash F(z_k)\big)}
%       -\nabla_{d\lambda} L_g(z_k)\diag[\lambda_k\oslash F(z_k)]\nabla_z F(z_k)\Delta z_s 
%       +\mu \nabla_{d\lambda} L_g(z_k)\big(\mbf 1\oslash F(z_k)\big)\\
%         G(z_k)}
%       \eqv \\
%       \matt{
%         \nabla_{uz} L_f(z_k,\lambda_k,\nu_k) -\nabla_{u\lambda} L_f(z_k)\diag[\lambda_k\oslash F(z_k)]\nabla_z F(z_k)&\nabla_{u\nu} L_f(z_k)\\
%         \nabla_{dz} L_g(z_k,\lambda_k,\nu_k) -\nabla_{d\lambda} L_g(z_k)\diag[\lambda_k\oslash F(z_k)]\nabla_z F(z_k)&\nabla_{d\nu} L_g(z_k)\\
%         \nabla_z G(z_k) &0}
%       \matt{\Delta z_s\\\Delta\nu_s}\\
%       =-\matt{
%         \nabla_u L_f(z_k,\lambda_k,\nu_k)
% -\nabla_{u\lambda} L_f(z_k)\lambda_k
%       -{\color{blue}\nabla_{u\lambda} L_f(z_k)\big((\nabla_zF(z_k) \Delta z_a)\odot\Delta\lambda_a\oslash F(z_k)\big)}
%       +\mu \nabla_{u\lambda} L_f(z_k)\big(\mbf 1\oslash F(z_k)\big)\\
%         \nabla_d L_g(z_k,\lambda_k,\nu_k)
% -\nabla_{d\lambda} L_g(z_k)\lambda_k
%       -{\color{blue}\nabla_{d\lambda} L_g(z_k)\big((\nabla_zF(z_k) \Delta z_a)\odot\Delta\lambda_a\oslash F(z_k)\big)}
%       +\mu \nabla_{d\lambda} L_g(z_k)\big(\mbf 1\oslash F(z_k)\big)\\
%         G(z_k)}
      \eqv \\
      \matt{
        \nabla_{uz} L_f(z_k,\lambda_k,\nu_k) -\nabla_{u\lambda} L_f(z_k)\diag[\lambda_k\oslash F(z_k)]\nabla_z F(z_k)&\nabla_{u\nu} L_f(z_k)\\
        \nabla_{dz} L_g(z_k,\lambda_k,\nu_k) -\nabla_{d\lambda} L_g(z_k)\diag[\lambda_k\oslash F(z_k)]\nabla_z F(z_k)&\nabla_{d\nu} L_g(z_k)\\
        \nabla_z G(z_k) &0}
      \matt{\Delta z_s\\\Delta\nu_s}\\
      =-\matt{
        \nabla_u f(z_k)+\nabla_u \big((\nu_{fu})_k G_u(z_k)\big)
        -{\color{blue}\nabla_{u\lambda} L_f(z_k)\big((\nabla_zF(z_k) \Delta z_a)\odot\Delta\lambda_a\oslash F(z_k)\big)}
        +\mu \nabla_{u\lambda} L_f(z_k)\big(\mbf 1\oslash F(z_k)\big)\\
        \nabla_d g(z_k)+\nabla_d \big((\nu_{gd})_k G_d(z_k)\big)
        -{\color{blue}\nabla_{d\lambda} L_g(z_k)\big((\nabla_zF(z_k) \Delta z_a)\odot\Delta\lambda_a\oslash F(z_k)\big)}
        +\mu \nabla_{d\lambda} L_g(z_k)\big(\mbf 1\oslash F(z_k)\big)\\
        G(z_k)}.
    \end{multline*}


\item Update the estimates along the search direction so that the
  inequalities in \eqref{eq:primal-dual-ud-inequalities} hold strictly:
  \begin{align*}
    z_{k+1}&=z_k+\alpha_s\Delta z_s, &
    \nu_{k+1}&=\nu_k+\alpha_s \Delta \nu_s, &
    \lambda_{k+1}&=\lambda_k+\alpha_s \Delta \lambda_s
  \end{align*}
  where
  \begin{align*}
    \alpha_s&\eqdef\min\{\alpha_\mrm{primal},\alpha_\mrm{dual}\}, 
  \end{align*}
  and
  \begin{align}
    \alpha_\mrm{primal}&\eqdef \max\Big\{\alpha\in[0,1]:F(z_k+\frac{\alpha}{.99} \Delta z_s)\ge 0\Big\}, &
    \alpha_\mrm{dual}&\eqdef \max\Big\{\alpha\in[0,1]:\lambda_k+\frac{\alpha}{.99} \Delta\lambda_s\ge 0 \Big\}.
  \end{align}

\item Repeat from \ref{en:loop} with an incremented value for $k$ until
  \begin{align}\label{eq:termination-ud}
    \|\nabla_u L_f(z_k,\lambda_k,\nu_k)\|&\le \epsilon_u, &
    \|\nabla_d L_g(z_k,\lambda_k,\nu_k)\|&\le \epsilon_d, &
    \|G(z_k)\|&\le \epsilon_G, &
    \lambda_k 'F(z_k)&\le \epsilon_\mrm{gap}.
  \end{align}
  for sufficiently small tolerances $\epsilon_u$, $\epsilon_d$, $\epsilon_G$,
  $\epsilon_\mrm{gap}$. \frqed
\end{steps}
\end{algorithm}

When the functions $L_f$ and $L_g$ that appear in the unconstrained
minimizations in \eqref{eq:central-points-ud} have a single stationary
point that corresponds to their global minimum, termination of the
Algorithm~\ref{al:primal-dual} guarantees that the assumptions of
Lemma~\ref{le:equilibria-gap} hold [up to the tolerances in
\eqref{eq:termination-ud}] and we obtain the desired solution to
\eqref{eq:equilibria}.

\medskip

The desired uniqueness of the stationary point holds, e.g., when the
function $f(u,d)$ is convex in $u$, $g(u,d)$ is convex in $d$,
$F_u(u,d)$ is concave in $u$, $F_d(u,d)$ is concave in $d$, and
$G_u(u,d)$ is linear in $u$, and $G_d(u,d)$ is linear in $d$. However,
in practice the Algorithm~\ref{al:primal-dual} can find solutions to
\eqref{eq:equilibria} even when these convexity assumptions do not
hold. For problems for which one cannot be sure whether the
Algorithm~\ref{al:primal-dual} terminated at a global minimum of the
unconstrained problem, one may run several instances of the algorithm with
random initial conditions. Consistent results for the optimizations
across multiple initializations will provide an indication that a global
minimum has been found.


\begin{remark}[Smoothness]
  Algorithm~\ref{al:primal-dual} requires all the functions
  $f,g,F_u,F_d,G_u,G_d$ to be twice differentiable for the computation
  of the matrices that appear in
  \eqref{eq:equilibrium-linearization}. However, this does not
  preclude the use of this algorithm in many problems where these
  functions are not differentiable because it is often possible to
  re-formulate non-smooth optimizations into smooth ones by
  appropriate transformations that often introduce additional
  optimization variables. Common examples of these transformations
  include the minimization of criteria involving $\ell_p$ norms, such as
  the ``non-differentiable $\ell_1$ optimization''
  \begin{align*}
    \min \big\{ \|A_{m\times n}x-b\|_{\ell_1}+\cdots: x\in\R^n,\dots \big\}
  \end{align*}
  which is equivalent to the following ``smooth optimization''
  \begin{align*}
    \min \big\{v'\mbf 1_m +\cdots:x\in\R^n,v\in\R^m, -v\le Ax-b\le v,\dots\big\};
  \end{align*}
  or the ``non-differentiable $\ell_2$ optimization''
  \begin{align*}
    \min \big\{\|A_{m\times n}x-b\|_{\ell_2}+\cdots: x\in\R^n,\dots\big\}
  \end{align*}
  \begin{align*}
    \min \big\{v +\cdots:x\in\R^n,v\ge0, v^2\ge (Ax-b)'(Ax-b),\dots \big\}.
  \end{align*}
  More examples of such transformations can be found, e.g., in
  \cite{GrantBoyd2008}.\joao{We should dig for more references.} \frqed
\end{remark}

\section{Interior-Point Method for Minimax Problems with a Common Latent Variables}
\label{sec:ipm-latent}

Like in Section~\ref{sec:ipm}, our goal is to find a pair
$(u^*,d^*)\in\scr{U}[d^*]\times\scr{D}[u^*]$ that simultaneously solves the
two coupled optimizations
\begin{align}\label{eq:equilibria-bar}
  \bar f(u^*,d^*)&=\min_{u\in\bar{\scr{U}}[d^*]} \bar f(u,d^*), &
  \bar g(u^*,d^*)&=\min_{d\in\bar{\scr{D}}[u^*]} \bar g(u^*,d),
\end{align}
with
\begin{subequations}
  \label{eq:U-D-bar}
  \begin{align}
    \bar{\scr{U}}[d]&\eqdef\{u\in\R^{N_u}: \bar F_u(u,d)\ge 0,\; \bar G_u(u,d)=0\}, \\
    \bar{\scr{D}}[u]&\eqdef\{d\in\R^{N_d}: \bar F_d(u,d)\ge 0,\; \bar G_d(u,d)=0\},
  \end{align}
\end{subequations}
for given functions $\bar f:\R^{N_u}\times\R^{N_d}\in\R$, $\bar
g:\R^{N_u}\times\R^{N_d}\in\R$, $\bar F_u:\R^{N_u}\times\R^{N_d}\to\R^{M_u}$, $\bar
F_d:\R^{N_u}\times\R^{N_d}\to\R^{M_d}$, $\bar G_u:\R^{N_u}\times\R^{N_d}\to\R^{K_u}$,q
$\bar G_d:\R^{N_u}\times\R^{N_d}\to\R^{K_d}$. However, we are now interested in
cases where these functions can be expressed in terms of common
\emph{latent variables}. Specifically, these functions can be
expressed as
\begin{align*}
  \bar f(u,d)&=f\big(u,d,\chi(u,d)\big), &
  \bar g(u,d)&=g\big(u,d,\chi(u,d)\big), \\
  \bar F_u(u,d)&=F_u\big(u,d,\chi(u,d)\big), &
  \bar G_u(u,d)&=G_u\big(u,d,\chi(u,d)\big), \\
  \bar F_d(u,d)&=F_d\big(u,d,\chi(u,d)\big), & \bar
  G_d(u,d)&=G_d\big(u,d,\chi(u,d)\big),
\end{align*}
$\forall u\in\R^{N_u},d\in\R^{N_d}$, for a function $\chi:\R^{N_u}\times\R^{N_d}\to\R^{N_x}$
whose value $\chi(u,d)$ is defined implicitly by a function\joao{To
  eventually get square system of equations, we should have $K_x=N_x$.}
$H:\R^{N_u}\times\R^{N_d}\times\R^{N_x}\to\R^{K_x}$ and an equation of the form
\begin{align}\label{eq:H}
  H(u,d,x)=0.
\end{align}
The function $H$ is assumed to be such that \eqref{eq:H} has a unique
solution $x$ for every $u\in\R^{N_u}$, $d\in\R^{N_d}$. The followin
corollary of Lemma~\ref{le:equilibria-gap} is useful in situations
where it is difficult (or impossible) to find an explicit form for
$\chi$.

\subsection{Primal-dual method}

The following duality-like result provides the motivation for a
primal-dual-like method to solve the coupled minimizations in
\eqref{eq:equilibria-bar}. It provides a set of conditions, involving
two unconstrained optimizations, that provide an approximation to the
solution of \eqref{eq:equilibria-bar}. It improves upon a direct
application of Lemma~\ref{le:equilibria-gap} to
\eqref{eq:equilibria-bar}, in the implictely defined function $\chi$ does
not appear in the conditions.



\begin{corollary}[Approximate equilibrium]\label{co:equilibria-gap}
  Consider the coupled optimizations in \eqref{eq:equilibria-bar} and
  assume that for every $u\in\R^{N_u}$, $d\in\R^{N_d}$, the equation
  \eqref{eq:H} has a unique solution $x$.  Suppose that we have found
  \emph{primal variables} $\hat u\in\R^{N_u},\hat d\in\R^{N_d},\hat
  x\in\R^{N_x}$ and \emph{dual variables}
  $\hat\nu_{fu}\in\R^{K_u},\hat\nu_{fx}\in\R^{K_x},\hat\nu_{gd}\in\R^{K_d},\hat\nu_{gx}\in\R^{K_x}$
  that simultaneously satisfy all of the following conditions
  \begin{subequations}
    \begin{align}
      \label{eq:equality-constraints-ud-latent}
      &G_u(\hat u,\hat d,\hat x)=0, \qquad
      G_d(\hat u,\hat d,\hat x)=0, \qquad H(\hat u,\hat d,\hat x)=0, \\
      \label{eq:inequality-constraints-ud-latent}
      &\hat\lambda_{fu}\dot\ge0, \qquad \hat\lambda_{gd} \dot\ge 0, \qquad F_u(\hat u,\hat d,\hat x)\ge0,
      \qquad F_d(\hat u,\hat d,\hat x)\ge0,\\
      \label{eq:central-points-ud-latent-1}
      &L_f(\hat u,\hat d,\hat x,\hat\lambda_{fu},\hat\nu_{fu},\hat\nu_{fx})
      =\min_{u\in\R^{N_u},x\in\R^{N_x}}\; L_f(u,\hat d,x,\hat \lambda_{fu},\hat \nu_{fu},\hat\nu_{fx}), \\
      \label{eq:central-points-ud-latent-2}
      &L_g(\hat u,\hat d,\hat x,\hat\lambda_{gd},\hat\nu_{gd},\hat\nu_{gx})
      =\min_{d\in\R^{N_d},x\in\R^{N_x}}\; L_g(\hat u,d,x,\hat \lambda_{gd},\hat \nu_{gd},\hat\nu_{gx})
    \end{align}
  \end{subequations}
  where 
  \begin{align*}
    &L_f(u,d,x,\lambda_{fu},\nu_{fu},\nu_{fx})\eqdef f(u,d,x)-\lambda_{fu} F_u(u,d,x)+\nu_{fu} G_u(u,d,x)+\nu_{fx}H(u,d,x), \\
    &L_g(u,d,x,\lambda_{gd},\nu_{gd},\nu_{gx})\eqdef g(u,d,x)-\lambda_{gd} F_d(u,d,x)+\nu_{gd} G_d(u,d,x)+\nu_{gx}H(u,d,x).
  \end{align*}
  Then $(\hat u,\hat d)$ approximately satisfy \eqref{eq:equilibria-bar}
  in the sense that
  \begin{align}\label{eq:epsilon-equilibria-latent}
    \bar f(\hat u,\hat d)&\le\epsilon_f+\min_{u\in\bar{\scr{U}}[\hat d]} \bar f(u,\hat d), &
    \bar g(\hat u,\hat d)&\le\epsilon_g+\min_{d\in\bar{\scr{D}}[\hat u]} \bar g(\hat u,d),
  \end{align}
  with
  \begin{align*}
    \epsilon_f&\eqdef \hat\lambda_{fu} F_u(\hat u,\hat d,\hat x), &
    \epsilon_g&\eqdef \hat\lambda_{gd} F_d(\hat u,\hat d,\hat x).
    \tag*\frqed
  \end{align*}
\end{corollary}

Note that while Corollary~\ref{co:equilibria-gap} utilizes a single
primal (latent) variable $\hat x$, it requires two dual variables
$\hat\nu_{fx},\hat\nu_{gx}\in\R^{K_x}$ associated with the equality
constraint $H(\hat u,\hat d,\hat x)=0$.

\begin{proof-corollary}{\ref{co:equilibria-gap}}
  Since the equation in \eqref{eq:H} has a unique solution in $x$, the
  optimizations in \eqref{eq:equilibria-bar}, can be re-written as
  \begin{align}\label{eq:equilibria-x}
    \bar f(u^*,d^*)&=\min_{(u,x)\in\scr{U}[d^*]} f(u,d^*,x), &
    \bar g(u^*,d^*)&=\min_{(d,z)\in\scr{D}[u^*]} g(u^*,d,z),
  \end{align}
  with
  \begin{align}
    \scr{U}[d]&\eqdef\{(u,x)\in\R^{N_u}\times\R^{N_x}: F_u(u,d,x)\ge 0,\; G_u(u,d,x)=0,\;H(u,d,x)=0\}, \\
    \scr{D}[u]&\eqdef\{(d,z)\in\R^{N_d}\times\R^{N_x}: F_d(u,d,z)\ge 0,\; G_d(u,d,z)=0,\;H(u,d,z)=0\},
  \end{align}
  which is again of the form considered in Section~\ref{sec:ipm}, but
  for optimization variables $(u,x)$ and $(d,z)$ in higher dimensional
  spaces.
  
  \medskip
  
  Applying Lemma~\ref{le:equilibria-gap} to the new formulation in
  \eqref{eq:equilibria-x}, we conclude that if we find \emph{primal
    variables} $\hat u\in\R^{N},\hat x\in\R^{N},\hat d\in\R^{N_d},\hat
  z\in\R^{N_x}$ and \emph{dual variables}
  $\hat\lambda_{fu}\in\R^{M_u},\hat\lambda_{gd}\in\R^{M_d}$,
  $\hat\nu_{fu}\in\R^{K_u},\hat\nu_{fx}\in\R^{K_x},\hat\nu_{gd}\in\R^{K_d},\hat\nu_{gx}\in\R^{K_x}$
  that simultaneously satisfy all of the following conditions
\begin{subequations}
  \begin{align}
    \label{eq:equality-constraints-ud-xz}
    &G_u(\hat u,\hat d,\hat x)=0, \qquad H(\hat u,\hat d,\hat x)=0, \qquad
    G_d(\hat u,\hat d,\hat z)=0, \qquad H(\hat u,\hat d,\hat z)=0, \\
    \label{eq:inequality-constraints-ud-lat}
    &\hat\lambda_{fu}\dot\ge0, \qquad \hat\lambda_{gd} \dot\ge 0, \qquad F_u(\hat u,\hat d,\hat x)\ge0,
    \qquad F_d(\hat u,\hat d,\hat z)\ge0,\\
    \label{eq:central-points-ud-lat}
    &L_f(\hat u,\hat x,\hat d,\hat z,\hat\lambda_{fu},\hat\nu_{fu},\hat\nu_{fx})
    =\min_{u\in\R^{N_u},x\in\R^{N_x}}\; L_f(u,x,\hat d,\hat z,\hat \lambda_{fu},\hat \nu_{fu},\hat\nu_{fx}), \\
    &L_g(\hat u,\hat x,\hat d,\hat z,\hat\lambda_{gd},\hat\nu_{gd},\hat\nu_{gx})
    =\min_{d\in\R^{N_d},z\in\R^{N_x}}\; L_g(\hat u,\hat x,d,z,\hat \lambda_{gd},\hat \nu_{gd},\hat\nu_{gx})
  \end{align}
\end{subequations}
where 
\begin{align*}
  &L_f(u,x,d,z,\lambda_{fu},\nu_{fu},\nu_{fx})\eqdef f(u,d,x)-\lambda_{fu} F_u(u,d,x)+\nu_{fu} G_u(u,d,x)+\nu_{fx}H(u,d,x), \\
  &L_g(u,x,d,z,\lambda_{gd},\nu_{gd},\nu_{gx})\eqdef g(u,d,z)-\lambda_{gd} F_d(u,d,z)+\nu_{gd} G_d(u,d,z)+\nu_{gx}H(u,d,z); 
\end{align*}
then 
\begin{align*}
  \bar f(\hat u,\hat d)=f(\hat u,\hat d,\hat x)
  &\le\epsilon_f+\min_{(u,x)\in\scr{U}[\hat d]} f(u,\hat d,x)
  =\epsilon_f+\min_{u\in\bar{\scr{U}}[\hat d]} \bar f(u,\hat d),\\
  \bar g(\hat u,\hat d)=g(\hat u,\hat d,\hat z)
  &\le\epsilon_g+\min_{(d,z)\in\scr{D}[\hat u]} g(\hat u,d,z)
  =\epsilon_g+\min_{d\in\bar{\scr{D}}[\hat u]} \bar g(\hat u,d)
\end{align*}
with
\begin{align*}
  \epsilon_f&\eqdef \hat\lambda_{fu} F_u(\hat u,\hat d,\hat x), &
  \epsilon_g&\eqdef \hat\lambda_{gd} F_d(\hat u,\hat d,\hat z).
  \tag*\frqed
\end{align*}
The result follows from this, together with the observation that $\hat
x=\hat z$ because the equations $H(\hat u,\hat d,\hat x)=0$ and
$H(\hat u,\hat d,\hat z)=0$ in \eqref{eq:equality-constraints-ud-xz}
must have exactly the same solution $\hat x=\hat z$.\frQED
\end{proof-corollary}

\subsection{Interior-point primal-dual equilibria algorithm}

The method proposed consists of using Newton iterations to solve a
system of nonlinear equations on the primal variables $\hat
u\in\R^{N_u},\hat d\in\R^{N_d},\hat x\in\R^{N_x}$ and dual variables
$\hat\lambda_{fu}\in\R^{M_u},\hat\lambda_{gd}\in\R^{M_d}$,
$\hat\nu_{fu}\in\R^{K_u},\hat\nu_{gd}\in\R^{K_d},\hat\nu_{fx},\hat\nu_{gx}\in\R^{K_x}$ introduced in
Corollary~\ref{co:equilibria-gap}. The specific system of equations
consists of:
\begin{enumerate}
\item the first-order optimality conditions for the unconstrained
  minimizations in
  \eqref{eq:central-points-ud-latent-1}--\eqref{eq:central-points-ud-latent-2}\footnote{Given
    an integer $M$, we denote by $\mbf 0_M$ and by $\mbf 1_M$ the
    $M$-vectors with all entries equal to 0 and 1, respectively.}:
  \begin{subequations}\label{eq:equilibrium-latent}
    \begin{align}
      \nabla_u L_f(\hat u,\hat d,\hat
      x,\hat\lambda_{fu},\hat\nu_{fu},\hat\nu_{fx})&=\mbf 0_{N_u}, &
      \nabla_x L_f(\hat u,\hat d,\hat x,\hat\lambda_{fu},\hat\nu_{fu},\hat\nu_{fx})&=\mbf 0_{N_u}, \\
      \nabla_d L_g(\hat u,\hat d,\hat
      x,\hat\lambda_{gd},\hat\nu_{gd},\hat\nu_{gx})&=\mbf 0_{N_d}, & \nabla_x
      L_g(\hat u,\hat d,\hat x,\hat\lambda_{gd},\hat\nu_{gd},\hat\nu_{gx})&=\mbf
      0_{N_d}; &
    \end{align}
  \end{subequations}
\item the equality conditions \eqref{eq:equality-constraints-ud-latent}; and
\item the equations\footnote{Given two vectors $x,y\in\R^n$ we denote by
    $x\odot y\in\R^n$ and by $x\oslash y\in\R^n$ the entry-wise product and
    division of the two vectors, respectively.}
  \begin{align}\label{eq:barrier-ud-latent}
    F_u(\hat u,\hat d,\hat x)\odot\hat\lambda_{fu}&=\mu\mbf 1_{M_u}, &
    F_d(\hat u,\hat d,\hat x)\odot\hat\lambda_{gd}&=\mu\mbf 1_{M_d}, &
  \end{align}
  for some $\mu>0$, which leads to
  \begin{align*}
    \epsilon_f&=M_u \mu, & \epsilon_g&=M_d \mu.
  \end{align*}
\end{enumerate}
Since our goal is to find primal variables $\hat u,\hat d,\hat x$ for which
\eqref{eq:epsilon-equilibria-latent} holds with $\epsilon_f=\epsilon_g=0$, we shall make
the variable $\mu$ converge to zero as the Newton iterations
progress. This is done in the context of an interior-point method,
meaning that all variables will be initialized so that the inequality
constraints \eqref{eq:inequality-constraints-ud-lat} hold and the progression
along the Newton direction at each iteration will be selected so that
these constraints are never violated.

\medskip

The specific steps of the algorithm that follows are inspired by the
primal-dual interior-point method for a single optimization, as
described in \cite{Vandenberghe2010}. To describe this algorithm, we
define
\begin{align*}
  z&\eqdef\matt{\hat u\\\hat d\\\hat x}, &
  \lambda&\eqdef\matt{\hat\lambda_{fu}\\\hat\lambda_{gd}}, &
  \nu&\eqdef\matt{\hat\nu_{fu}\\\hat\nu_{fx}\\\hat\nu_{gd}\\\hat\nu_{gx}}, &
  G(z)&\eqdef\matt{G_u(\hat u,\hat d,\hat x)\\ G_d(\hat u,\hat d,\hat x)\\H(\hat u,\hat d,\hat x)}, &
  F(z)&\eqdef\matt{F_u(\hat u,\hat d)\\ F_d(\hat u,\hat d)}, &
\end{align*}
which allow us to re-write \eqref{eq:equilibrium-latent},
\eqref{eq:equality-constraints-ud-latent}, and \eqref{eq:barrier-ud-latent} as
\begin{subequations}
  \begin{gather}
    \label{eq:primal-dual-ud-equalities-latent-1}
    \nabla_u L_f(z,\lambda,\nu)=\mbf 0_{N_u}, \qquad
    \nabla_x L_f(z,\lambda,\nu)=\mbf 0_{N_x}, \\
    \nabla_d L_g(z,\lambda,\nu)=\mbf 0_{N_d}, \qquad
    \nabla_x L_g(z,\lambda,\nu)=\mbf 0_{N_x}, \\
    \label{eq:primal-dual-ud-equalities-latent-2}
    G(z)=\mbf 0 _{K_u+K_d+K_x}, \qquad
    \lambda\odot F(z)=\mu\mbf 1_{M_u+M_d}, 
  \end{gather}
and \eqref{eq:inequality-constraints-ud-latent} as
\begin{gather}
  \label{eq:primal-dual-ud-inequalities-latent}
  \lambda\ge \mbf 0_{M_u+M_d}, \qquad F(z)\ge \mbf 0_{M_u+M_d}.
\end{gather}
\end{subequations}

\clearpage

\begin{algorithm}[Primal-dual optimization with common latent variable]~\label{al:primal-dual-latent}
\begin{steps}
\item Start with estimates $z_0,\lambda_0,\nu_0$ that satisfy the inequalities
  $\lambda_0\ge 0$, $F(z_0)\ge 0$ in \eqref{eq:primal-dual-ud-inequalities-latent} and set $k=0$.

  It is often a good idea to start with a value for $z_0$ that
  satisfies the equality constraint $G(z_0)=0$, and
  \begin{align*}
    \lambda_0=\mu\mbf 1_{M_u+M_d}\oslash F(z_0),
  \end{align*}
  which guarantees that we initially have
  $\lambda_0\odot F(u_0)=\mu\mbf 1_{M_u+M_d}$.\joao{Starting with
    $\nu_0=0$ seems to be okay. Note that the value of $\nu$ only affects
    the matrix in the left-hand side of
    \eqref{eq:equilibrium-linearization-31} when $G(u)$ is nonlinear.}


\item Linearize the equations in
  \eqref{eq:primal-dual-ud-equalities-latent-1}--\eqref{eq:primal-dual-ud-equalities-latent-2}
  around a current estimate $z_k,\lambda_k,\nu_k$, leading to
  % \begin{align*}
  %   \matt{
  %     \nabla_{uz} L_f(z_k,\lambda_k,\nu_k)&\nabla_{u\nu} L_f(z_k)&\nabla_{u\lambda} L_f(z_k)\\
  %     \nabla_{dz} L_g(z_k,\lambda_k,\nu_k)&\nabla_{d\nu} L_g(z_k)&\nabla_{d\lambda} L_g(z_k)\\
  %     \nabla_z G(z_k) &0&0\\
  %     \diag(\lambda_k)\nabla_zF(z_k)&0&\diag[F(z_k)] }
  %   \matt{\Delta z\\\Delta\nu\\\Delta\lambda} =-\matt{
  %     \nabla_u L_f(z_k,\lambda_k,\nu_k)\\
  %     \nabla_d L_g(z_k,\lambda_k,\nu_k) \\
  %     G(z_k)\\
  %     F(z_k)\odot\lambda_k-\mu\mbf 1 }
  % \end{align*}
  % or equivalently\joao{This equation may not be needed.}
  \begin{align*}
    \matt{
      \nabla_{uz} L_f(z_k,\lambda_k,\nu_k)&\nabla_{u\nu} L_f(z_k)&\nabla_{u\lambda} L_f(z_k)\\
      \nabla_{xz} L_f(z_k,\lambda_k,\nu_k)&\nabla_{x\nu} L_f(z_k)&\nabla_{x\lambda} L_f(z_k)\\
      \nabla_{dz} L_g(z_k,\lambda_k,\nu_k)&\nabla_{d\nu} L_g(z_k)&\nabla_{d\lambda} L_g(z_k)\\
      \nabla_{xz} L_g(z_k,\lambda_k,\nu_k)&\nabla_{x\nu} L_g(z_k)&\nabla_{x\lambda} L_g(z_k)\\
      \nabla_z G(z_k) &0&0\\
      -\nabla_zF(z_k)&0&-\diag[F(z_k)\oslash \lambda_k] }
    \matt{\Delta z\\\Delta\nu\\\Delta\lambda} =-\matt{
      \nabla_u L_f(z_k,\lambda_k,\nu_k)\\
      \nabla_x L_f(z_k,\lambda_k,\nu_k)\\
      \nabla_d L_g(z_k,\lambda_k,\nu_k) \\
      \nabla_x L_g(z_k,\lambda_k,\nu_k) \\
      G(z_k)\\
      -F(z_k)+\mu \mbf 1\oslash \lambda_k }.
  \end{align*}
  Since the vectors $F(z_k), \lambda_k$ have positive entries, we can
  solve this system of equations by first eliminating
  \begin{multline*}
    \nabla_z F(z_k)\Delta z
    +\diag[F(z_k)\oslash\lambda_k]\Delta\lambda
    =-F(z_k)+\mu \mbf 1\oslash \lambda_k \eqv\\
    \Delta\lambda
    =-\lambda_k
    -\diag[\lambda_k\oslash F(z_k)]\nabla_z F(z_k)\Delta z 
    +\mu \mbf 1\oslash F(z_k)
  \end{multline*}
  which leads to
    \begin{multline*}
      \matt{
        \nabla_{uz} L_f(z_k,\lambda_k,\nu_k)&\nabla_{u\nu} L_f(z_k)\\
        \nabla_{xz} L_f(z_k,\lambda_k,\nu_k)&\nabla_{x\nu} L_f(z_k)\\
        \nabla_{dz} L_g(z_k,\lambda_k,\nu_k)&\nabla_{d\nu} L_g(z_k)\\
        \nabla_{xz} L_g(z_k,\lambda_k,\nu_k)&\nabla_{x\nu} L_g(z_k)\\
        \nabla_z G(z_k) &0}
      \matt{\Delta z\\\Delta\nu}
      =-\matt{
        \nabla_u L_f(z_k,\lambda_k,\nu_k)+\nabla_{u\lambda} L_f(z_k)\Delta\lambda\\
        \nabla_x L_f(z_k,\lambda_k,\nu_k)+\nabla_{x\lambda} L_f(z_k)\Delta\lambda\\
        \nabla_d L_g(z_k,\lambda_k,\nu_k)+\nabla_{d\lambda} L_g(z_k)\Delta\lambda\\
        \nabla_x L_g(z_k,\lambda_k,\nu_k)+\nabla_{x\lambda} L_g(z_k)\Delta\lambda\\
        G(z_k)}
      \eqv \\
      \matt{
        \nabla_{uz} L_f(z_k,\lambda_k,\nu_k) -\nabla_{u\lambda} L_f(z_k)\diag[\lambda_k\oslash F(z_k)]\nabla_z F(z_k)&\nabla_{u\nu} L_f(z_k)\\
        \nabla_{xz} L_f(z_k,\lambda_k,\nu_k) -\nabla_{x\lambda} L_f(z_k)\diag[\lambda_k\oslash F(z_k)]\nabla_z F(z_k)&\nabla_{x\nu} L_f(z_k)\\
        \nabla_{dz} L_g(z_k,\lambda_k,\nu_k) -\nabla_{d\lambda} L_g(z_k)\diag[\lambda_k\oslash F(z_k)]\nabla_z F(z_k)&\nabla_{d\nu} L_g(z_k)\\
        \nabla_{xz} L_g(z_k,\lambda_k,\nu_k) -\nabla_{x\lambda} L_g(z_k)\diag[\lambda_k\oslash F(z_k)]\nabla_z F(z_k)&\nabla_{x\nu} L_g(z_k)\\
        \nabla_z G(z_k) &0}
      \matt{\Delta z\\\Delta\nu}\\
      =-\matt{
        \nabla_u L_f(z_k,\lambda_k,\nu_k)-\nabla_{u\lambda} L_f(z_k)\lambda_k
        +\mu \nabla_{u\lambda} L_f(z_k)\big(\mbf 1\oslash F(z_k)\big)\\
        \nabla_x L_f(z_k,\lambda_k,\nu_k)-\nabla_{x\lambda} L_f(z_k)\lambda_k
        +\mu \nabla_{x\lambda} L_f(z_k)\big(\mbf 1\oslash F(z_k)\big)\\
        \nabla_d L_g(z_k,\lambda_k,\nu_k)-\nabla_{d\lambda} L_g(z_k)\lambda_k
        +\mu \nabla_{d\lambda} L_g(z_k)\big(\mbf 1\oslash F(z_k)\big)\\
        \nabla_x L_g(z_k,\lambda_k,\nu_k)-\nabla_{x\lambda} L_g(z_k)\lambda_k
        +\mu \nabla_{x\lambda} L_g(z_k)\big(\mbf 1\oslash F(z_k)\big)\\
        G(z_k)}
      % \eqv \\
      % \matt{
      %   \nabla_{uz} L_f(z_k,\lambda_k,\nu_k) -\nabla_{u\lambda} L_f(z_k)\diag[\lambda_k\oslash F(z_k)]\nabla_z F(z_k)&\nabla_{u\nu} L_f(z_k)\\
      %   \nabla_{dz} L_g(z_k,\lambda_k,\nu_k) -\nabla_{d\lambda} L_g(z_k)\diag[\lambda_k\oslash F(z_k)]\nabla_z F(z_k)&\nabla_{d\nu} L_g(z_k)\\
      %   \nabla_z G(z_k) &0}
      % \matt{\Delta z\\\Delta\nu}\\
      % =-\matt{
      %   \nabla_u \big(L_f(z_k,\lambda_k,\nu_k)-\nabla_{\lambda} L_f(z_k)\lambda_k\big)
      %   +\mu \nabla_{u\lambda} L_f(z_k)\big(\mbf 1\oslash F(z_k)\big)\\
      %   \nabla_d \big(L_g(z_k,\lambda_k,\nu_k)-\nabla_{\lambda} L_g(z_k)\lambda_k\big)
      %   +\mu \nabla_{d\lambda} L_g(z_k)\big(\mbf 1\oslash F(z_k)\big)\\
      %   G(z_k)}
      % \eqv \\
      % \matt{
      %   \nabla_{uz} L_f(z_k,\lambda_k,\nu_k) -\nabla_{u\lambda} L_f(z_k)\diag[\lambda_k\oslash F(z_k)]\nabla_z F(z_k)&\nabla_{u\nu} L_f(z_k)\\
      %   \nabla_{dz} L_g(z_k,\lambda_k,\nu_k) -\nabla_{d\lambda} L_g(z_k)\diag[\lambda_k\oslash F(z_k)]\nabla_z F(z_k)&\nabla_{d\nu} L_g(z_k)\\
      %   \nabla_z G(z_k) &0}
      % \matt{\Delta z\\\Delta\nu}\\
      % =-\matt{
      %   \nabla_u \big(f(z_k)+(\nu_{fu})_kG_u(z_k)\big)
      %   +\mu \nabla_{u\lambda} L_f(z_k)\big(\mbf 1\oslash F(z_k)\big)\\
      %   \nabla_d \big(g(z_k)+(\nu_{gd})_kG_d(z_k)\big)
      %   +\mu \nabla_{d\lambda} L_g(z_k)\big(\mbf 1\oslash F(z_k)\big)\\
      %   G(z_k)}
      \eqv \\
      \matt{
        \nabla_{uz} L_f(z_k,\lambda_k,\nu_k) -\nabla_{u\lambda} L_f(z_k)\diag[\lambda_k\oslash F(z_k)]\nabla_z F(z_k)&\nabla_{u\nu} L_f(z_k)\\
        \nabla_{xz} L_f(z_k,\lambda_k,\nu_k) -\nabla_{x\lambda} L_f(z_k)\diag[\lambda_k\oslash F(z_k)]\nabla_z F(z_k)&\nabla_{x\nu} L_f(z_k)\\
        \nabla_{dz} L_g(z_k,\lambda_k,\nu_k) -\nabla_{d\lambda} L_g(z_k)\diag[\lambda_k\oslash F(z_k)]\nabla_z F(z_k)&\nabla_{d\nu} L_g(z_k)\\
        \nabla_{xz} L_g(z_k,\lambda_k,\nu_k) -\nabla_{x\lambda} L_g(z_k)\diag[\lambda_k\oslash F(z_k)]\nabla_z F(z_k)&\nabla_{x\nu} L_g(z_k)\\
        \nabla_z G(z_k) &0}
      \matt{\Delta z\\\Delta\nu}\\
      =-\matt{
        \nabla_u \big(f(z_k)+(\nu_{fu})_k G_u(z_k)+(\nu_{fx})_k H(z_k)\big)
        +\mu \nabla_{u\lambda} L_f(z_k)\big(\mbf 1\oslash F(z_k)\big)\\
        \nabla_x \big(f(z_k)+(\nu_{fu})_k G_u(z_k)+(\nu_{fx})_k H(z_k)\big)
        +\mu \nabla_{x\lambda} L_f(z_k)\big(\mbf 1\oslash F(z_k)\big)\\
        \nabla_d \big(g(z_k)+(\nu_{gd})_k G_d(z_k)+(\nu_{gx})_k H(z_k)\big)
        +\mu \nabla_{d\lambda} L_g(z_k)\big(\mbf 1\oslash F(z_k)\big)\\
        \nabla_x \big(g(z_k)+(\nu_{gd})_k G_d(z_k)+(\nu_{gx})_k H(z_k)\big)
        +\mu \nabla_{x\lambda} L_g(z_k)\big(\mbf 1\oslash F(z_k)\big)\\
        G(z_k)}
      \eqv \\
      \matt{
        \nabla_{uz} L_f(z_k,\lambda_k,\nu_k) -\nabla_{u\lambda} L_f(z_k)\diag[\lambda_k\oslash F(z_k)]\nabla_z F(z_k)&
        \nabla_u G_u(z)'&\nabla_u H(z)'&0&0\\
        \nabla_{xz} L_f(z_k,\lambda_k,\nu_k) -\nabla_{x\lambda} L_f(z_k)\diag[\lambda_k\oslash F(z_k)]\nabla_z F(z_k)&
        \nabla_x G_u(z)'&\nabla_x H(z)'&0&0\\
        \nabla_{dz} L_g(z_k,\lambda_k,\nu_k) -\nabla_{d\lambda} L_g(z_k)\diag[\lambda_k\oslash F(z_k)]\nabla_z F(z_k)&
        0&0&\nabla_d G_d(z)'&\nabla_d H(z)'\\
        \nabla_{xz} L_g(z_k,\lambda_k,\nu_k) -\nabla_{x\lambda} L_g(z_k)\diag[\lambda_k\oslash F(z_k)]\nabla_z F(z_k)&
        0&0&\nabla_x G_d(z)'&\nabla_x H(z)'\\
        \nabla_z G_u(z_k) &0&0&0&0\\
        \nabla_z G_d(z_k) &0&0&0&0\\
        \nabla_z H(z_k) &0&0&0&0}
      \matt{\Delta z\\\Delta\nu}\\
      =-\matt{
        \nabla_u \big(f(z_k)+(\nu_{fu})_k G_u(z_k)+(\nu_{fx})_k H(z_k)\big)
        +\mu \nabla_{u\lambda} L_f(z_k)\big(\mbf 1\oslash F(z_k)\big)\\
        \nabla_x \big(f(z_k)+(\nu_{fu})_k G_u(z_k)+(\nu_{fx})_k H(z_k)\big)
        +\mu \nabla_{x\lambda} L_f(z_k)\big(\mbf 1\oslash F(z_k)\big)\\
        \nabla_d \big(g(z_k)+(\nu_{gd})_k G_d(z_k)+(\nu_{gx})_k H(z_k)\big)
        +\mu \nabla_{d\lambda} L_g(z_k)\big(\mbf 1\oslash F(z_k)\big)\\
        \nabla_x \big(g(z_k)+(\nu_{gd})_k G_d(z_k)+(\nu_{gx})_k H(z_k)\big)
        +\mu \nabla_{x\lambda} L_g(z_k)\big(\mbf 1\oslash F(z_k)\big)\\
        G_u(z_k)\\
        G_d(z_k)\\
        H(z_k)}
    \end{multline*}


\end{steps}
\end{algorithm}


%\bibliographystyle{ieeetr}
\bibliographystyle{abbrvnat}
\bibliography{strings,jph,crossrefs,optimization}

%\printindex

\end{document}

%%% Local Variables: 
%%% mode: latex
%%% eval: (tex-pdf-mode)  ; only for pdflatex
%%% TeX-master: t
%%% End: 

